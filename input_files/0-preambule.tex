%!TEX root = ../main.tex

%% ==============================================================
%% Encoding

\usepackage[french]{babel}
\usepackage[utf8]{inputenc}
\usepackage[T1]{fontenc}
\usepackage[autolanguage,np]{numprint}

%% ==============================================================
%% Title page style

%% INPT / Toulouse University title page
\usepackage[ED=MITT-SIAO, Ets=INP]{tlsflyleaf}

% \usepackage{lmodern}\normalfont
% \DeclareFontShape{T1}{lmr}{bx}{sc}{ <-> ssub * cmr/bx/sc }{}
% \DeclareOldFontCommand{\bf}{\normalfont\bfseries}{\mathbf}
% \usepackage[osf,sc]{mathpazo}

%% Setup basic string
\title{Reconstruction rapide d'images multi-bandes partiellement échantillonnées en spectromicroscopie EELS.}
%
\author{Etienne Monier}
%
\defencedate{9 oct. 2020}
%
\lab{Institut de Recherche en Informatique de Toulouse (UMR 5505)}

%% Boss
\nboss{3}
%
\makesomeone{boss}{1}{Nicolas \textsc{Dobigeon}}{Professeur des Universités, Toulouse INP}{Directeur de thèse}
\makesomeone{boss}{2}{Thomas \textsc{Oberlin}}{Enseignant-chercheur, ISAE-SUPAERO}{Co-directeur de thèse}
\makesomeone{boss}{3}{Nathalie \textsc{Brun}}{Chargé de Recherche CNRS, LPS }{Co-directrice de thèse}

%% Referee
\nreferee{2}
\makesomeone{referee}{1}{Cécile \textsc{Hébert}}{Professeur Associée, EPFL}{Rapporteur}
\makesomeone{referee}{2}{Vincent \textsc{Mazet}}{Maître de Conférences, Université de Strasbourg}{Rapporteur}

%% Judges
\njudge{8}
\makesomeone{judge}{1}{Cécile \textsc{Hébert}}{Professeur Associée, EPFL}{Rapporteur}
\makesomeone{judge}{2}{Vincent \textsc{Mazet}}{Maître de Conférences, Université de Strasbourg}{Rapporteur}
\makesomeone{judge}{3}{Jérôme \textsc{Idier}}{Directeur de Recherche CNRS, LS2N }{Examinateur}
\makesomeone{judge}{4}{Ferréol \textsc{Soulez}}{Astronome Adjoint, CRAL}{Examinateur}
\makesomeone{judge}{5}{Cyril \textsc{Ruckebush}}{Professeur des Universités, Université de Lille}{Examinateur}
\makesomeone{judge}{6}{Nicolas \textsc{Dobigeon}}{Professeur des Universités, Toulouse INP}{Co-directeur de thèse}
\makesomeone{judge}{7}{Thomas \textsc{Oberlin}}{Enseignant-chercheur, ISAE-SUPAERO}{Co-directeur de thèse}
\makesomeone{judge}{8}{Nathalie \textsc{Brun}}{Chargé de Recherche CNRS, LPS}{Co-directrice de thèse}

%% ==============================================================
%% Warning filtering

\usepackage{silence}
%\WarningFilter{biblatex}{Patching footnotes failed}
%\WarningFilter{biblatex}{Attempt to redefine deprecated}
\WarningFilter{latex}{Marginpar on page}
\WarningFilter{latexfont}{Font shape}
\WarningFilter{latexfont}{Some font}
\WarningFilter{latexfont}{Size substitutions}


%% ==============================================================
%% Hyperef links

\usepackage[svgnames]{xcolor}
\definecolor{mydarkblue}{rgb}{0,0.08,0.45}
\definecolor{myblue}{RGB}{18,75,126}
\definecolor{burgundy}{RGB}{128,0,32}




%% ==============================================================
%% Math packages

%% Math symbols and font
\usepackage{amsmath}    % Main commands.
\usepackage{amssymb}    % Main symbols
\usepackage{dsfont}     % \mathds{1} pour indicatrice
\usepackage{amsfonts}
\usepackage{amsthm}
\usepackage{mathrsfs}   % Ralph Smith’s Formal Script Font : mathscr{A}
\usepackage{mathtools}  % \DeclarePairedDelimiter{\ceil}{\lceil}{\rceil}
\usepackage{stmaryrd}   % \llbracket et \rrbracket %sinon : $[\![$ et $]\!]$

%% Theorems and definitions
\newtheorem{mydef}{Définition}


%% ==============================================================
%% Glossary

% Terminal: makeglossaries main

% Calculate Full width length.
\TufteRecalculate
\newlength\fullwidthwidth
\makeatletter\setlength\fullwidthwidth{\@tufte@fullwidth}\makeatother

% Calculate
\newlength\centralcol
\setlength\centralcol{\fullwidthwidth}
\addtolength{\centralcol}{-5cm}

% Glossaries Package call.
\usepackage[%
% nopostdot,     % If no final point is desired for description.
nonumberlist,  % The location should not be displayed.
acronym,       % Defines a default acronym glossary.
toc,           % This should appear in toc.
section=subsection,     % Sets the printglossaries be a section.
numberedsection=false,  % To have section* instead of section.
nogroupskip=true,
xindy,
ucmark,
shortcuts,
savewrites]{glossaries}

%
% New gloassaries
%

% Glossaries sections
% Adds a key to glossaries entries
\glsaddstoragekey{shape}{}{\glsshape}



%
% Format
%

% Defines the accronym display style. Long desc. first, then short one.
\setacronymstyle{long-short}

% Removes space after section name
\renewcommand{\glossarypreamble}{\vspace*{-\baselineskip}}

% This command disables hyperlinks from text to glossary list.
\glsdisablehyper

% https://tex.stackexchange.com/questions/269565/glossaries-how-to-customize-list-of-symbols-with-additional-column-for-units

\newglossarystyle{symbunitlong}{%
    \setglossarystyle{long3col}% base this style on the list style
    \renewenvironment{theglossary}{% Change the table type --> 3 columns
        \begin{longtable}{p{1cm}p{2cm}p{\centralcol}}}%
        {\end{longtable}}%
    \renewcommand*{\glossentry}[2]{%  Change the displayed items
        \glstarget{##1}{\glossentryname{##1}} %
        & %
        \ifglshasfield{shape}{##1}{$\in\mathbb{R}^{\glsshape{##1}}$}{}
        % Shape
        & \glossentrydesc{##1}% Description
        \tabularnewline
    }
}

\newglossarystyle{acrolong}{%
    \setglossarystyle{long3col}% base this style on the list style
    \renewenvironment{theglossary}{% Change the table type --> 3 columns
        \begin{longtable}{p{2cm}p{\fullwidthwidth-2cm}}}%
        {\end{longtable}}%
    %
    %    \renewcommand*{\glossaryheader}{%  Change the table header
    %        \bfseries Sign & \bfseries Description & \bfseries Unit \\
    %        \hline
    %        \endhead}
    \renewcommand*{\glossentry}[2]{%  Change the displayed items
        \glstarget{##1}{\glossentryname{##1}} %
        & \glossentrydesc{##1}% Description
        \tabularnewline
    }
}


%
% Starts glossaries and input entries
%

\newglossary[slg1]{notgen}{sls1}{slo1}{Notations générales}
\newglossary[slg2]{sets}{sls2}{slo2}{Ensembles}
\newglossary[slg3]{functions}{sls3}{slo3}{Fonctions}
\newglossary[slg4]{probas}{sls4}{slo4}{Probabilités}
\newglossary[slg5]{dimensions}{sls5}{slo5}{Dimensions}
\newglossary[slg6]{modele}{sls6}{slo6}{Modélisation du problème d'inpainting}
\newglossary[slg7]{demelange}{sls7}{slo7}{Démélange}
\newglossary[slg8]{acp}{sls8}{slo8}{Analyse en composantes principales}
\newglossary[slg9]{chap3}{sls9}{slo9}{Notations propres au chapitre 3}
\newglossary[slg10]{chap4}{sls10}{slo10}{Notations propres au chapitre 4}


\makeglossaries

%!TEX root = ../main.tex
% To build the glossary: makeglossaries main

%%%%% Acronyms
% \newacronym[plural={<plural acronym>},
%             first={<text displayed at first occurrence>},
%             firstplural={<idem, with plural>}]
%             {<label>}
%             {<acronym>}
%             {<full name to display in acronym section>}

% Laboratoires
\newacronym{irit}{IRIT}{Institut de Recherche en Informatique de Toulouse}

\newacronym{lps}{LPS}{Laboratoire de Physique des Solides}

% Microscopes
\newacronym{stem}{STEM}{\textit{scanning transmission electron microscope}}

\newacronym{tem}{TEM}{\textit{transmission electron microscopy}}

\newacronym{sem}{SEM}{\textit{scanning electron microscope}}

\newacronym{mfa}{AFM}{\textit{atomic force microscopy}}

\newacronym{irm}{IRM}{imagerie par résonance magnétique}

% Modalités
\newacronym{haadf}{HAADF}{\textit{high-angle annular dark-field}}

\newacronym{adf}{ADF}{\textit{annular dark-field}}

\newacronym{eels}{EELS}{\textit{electron energy loss spectroscopy}}

\newacronym{edx}{EDX}{\textit{energy-dispersive X-ray spectroscopy}}


% Techniques
\newacronym{acp}{ACP}{analyse en composantes principales}

\newacronym{aci}{ACI}{analyse en composantes indépendantes}

\newacronym{cs}{CS}{\textit{compressed sensing}}

\newacronym{ppv}{PPV}{plus proches voisins}

\newacronym{ad}{AD}{apprentissage de dictionnaire}

\newacronym{ebi}{EBI}{\textit{exemplar-based inpainting}}

\newacronym{mc}{MC}{moindres carrés}
        
\newacronym{dct}{DCT}{transformée en cosinus discrète}

\newacronym{tv}{TV}{variation totale}

\newacronym{goal}{GOAL}{\textit{geometric analysis operator learning}}

\newacronym{vca}{VCA}{\textit{vertex component analysis}}

\newacronym{sisal}{SISAL}{\textit{simplex identification via variable
    splitting and augmented lagrangian}}

\newacronym{sunsal}{SUNSAL}{\textit{spectral unmixing by splitting and augmented Lagrangian}}

\newacronym{map}{MAP}{maximum a posteriori}

\newacronym{mcmc}{MCMC}{Monte Carlo par chaîne de Markov}

\newacronym{svd}{SVD}{décomposition en valeurs singulières}


% \newacronym{}{}{}

% Métriques

\newacronym{snr}{SNR}{rapport signal-à-bruit}

\newacronym{nmse}{NMSE}{\textit{normalized mean square error}}

\newacronym{asad}{aSAD}{\textit{average spectral angle distance}}

\newacronym{ssim}{SSIM}{\textit{structural similarity}}

% Méthodes proposée

\newacronym{s2n}{S2N}{\textit{smoothed nuclear norm}}

\newacronym{3s}{3S}{\textit{smoothed subspace}}

\newacronym{cls}{CLS}{\textit{cosine least square}}

\newacronym{ista}{ISTA}{\textit{iterative shrinkage thresholding algorithm}}

\newacronym{fista}{FISTA}{\textit{fast iterative shrinkage thresholding algorithm}}

\newacronym{bpfa}{BPFA}{\textit{beta-process factor analysis}}

\newacronym{itkrmm}{ITKrMM}{\textit{iterative thresholding and K residual means for masked data}}

\newacronym{wksvd}{wKSVD}{\textit{weighted K-SVD}}







%% Notations génériques %%%%%%%%%%%%%%%%%%%%%%%%%%%%%%%%%%%%%%%%%%%%%%%%

\newglossaryentry{g-a}{
    type=notgen, 
    name={\ensuremath{a}}, 
    description={Scalaire}, 
    sort={01}}

\newglossaryentry{g-av}{type=notgen,
    name={\ensuremath{\mathbf{a}}},
    description={Vecteur colonne}, 
    sort={02}}

\newglossaryentry{gavi}{type=notgen, 
    name={\ensuremath{a_i}}, 
    description={$i^\text{ème}$ composante du vecteur \gls{g-av}}, 
    sort={03}}

\newglossaryentry{g-A}{type=notgen, 
    name={\ensuremath{\mathbf{A}}},
    description={Matrice},
    sort={04}}

\newglossaryentry{g-Aij}{type=notgen,
    name={\ensuremath{a_{i,j}}},
    description={Coefficient $(i, j)$ de la matrice \gls{g-A}},
    sort={05}}

\newglossaryentry{g-Aj}{type=notgen, 
    name={\ensuremath{\mathbf{a}_j}},
    description={$j^\text{ème}$ colonne de la matrice \gls{g-A}},
    sort={06}}

\newglossaryentry{g-Ai}{type=notgen, 
    name={\ensuremath{\mathbf{A}_{i, :}}},
    description={$i^\text{ème}$ ligne de la matrice \gls{g-A}},
    sort={07}}

\newglossaryentry{g-Amn}{type=notgen, 
    name={\ensuremath{\mathbf{A}_{m:n}}},
    description={Concaténation des colonnes de la matrice \gls{g-A} d'indices compris entre $m$ et $n$},
    sort={08}}

\newglossaryentry{g-AI}{type=notgen, 
    name={\ensuremath{\mathbf{A}_{\mathcal{E}}}},
    description={Concaténation des colonnes de la matrice \gls{g-A} indexées par l'ensemble $\mathcal{E}$},
    sort={09}}

\newglossaryentry{g-T}{type=notgen, 
    name={\ensuremath{(\cdot)^{T}}},
    description={Transposée},
    sort={10}}

\newglossaryentry{g-pm}{type=notgen, 
    name={\ensuremath{\mathbf{AB}}},
    description={Produit matriciel},
    sort={11}}

\newglossaryentry{g-pmh}{type=notgen, 
    name={\ensuremath{\mathbf{A\cdot B}}},
    description={Produit de Hadamard (terme à terme)},
    sort={12}}

\newglossaryentry{g-n1}{type=notgen, 
    name={\ensuremath{||\mathbf{a}||_1}},
    description={Norme $\ell_1$ de \gls{g-av}},
    sort={13}}

\newglossaryentry{g-n2}{type=notgen, 
    name={\ensuremath{||\mathbf{a}||_2}},
    description={Norme $\ell_2$ de \gls{g-av}},
    sort={14}}

\newglossaryentry{g-n12}{type=notgen, 
    name={\ensuremath{||\mathbf{A}||_{2, 1}}},
    description={Norme $\ell_{2, 1}$ de \gls{g-A}},
    sort={15}}

\newglossaryentry{g-nf}{type=notgen, 
    name={\ensuremath{||\mathbf{A}||_\mathrm{F}}},
    description={Norme de Frobenius de \gls{g-A}},
    sort={16}}

\newglossaryentry{g-nn}{type=notgen, 
    name={\ensuremath{||\mathbf{A}||_*}},
    description={Norme nucléaire de \gls{g-A}},
    sort={17}}

\newglossaryentry{g-zvec}{type=notgen, 
    name={\ensuremath{\mathbf{0}_n}},
    description={Vecteur nul de taille $n$},
    sort={18}}

\newglossaryentry{g-Id}{type=notgen, 
    name={\ensuremath{\mathds{1}_n}},
    description={Matrice identité de taille $n$},
    sort={18}}

%% Ensembles %%%%%%%%%%%%%%%%%%%%%%%%%%%%%%%%%%%%%%%%%%%%%%%%%%%%%%%%%%

%\newglossaryentry{e-N}{type=sets, 
%    name={\ensuremath{\mathbb{N}}},
%    description={Ensemble des nombres entiers naturels},
%    sort={1}}

\newglossaryentry{e-Nab}{type=sets, 
    name={\ensuremath{\llbracket m, n \rrbracket}},
    description={Ensemble des nombres entiers compris entre $m$ et $n$ inclus},
    sort={1}}

\newglossaryentry{e-R}{type=sets, 
    name={\ensuremath{\mathbb{R}}},
    description={Ensemble des nombres réels},
    sort={2}}

\newglossaryentry{e-Rn}{type=sets, 
    name={\ensuremath{\mathbb{R}^n}},
    description={Ensemble des vecteurs réels de taille $n$},
    sort={3}}

\newglossaryentry{e-Rmn}{type=sets, 
    name={\ensuremath{\mathbb{R}^{m \times n}}},
    description={Ensemble des matrices réelles de taille $m\times n$},
    sort={4}}

\newglossaryentry{e-boule}{type=sets, 
    name={\ensuremath{\mathcal{B}(x_0,r)}},
    description={Boule fermée en norme $\ell_2$ de centre $x_0$ et de rayon $r$},
    sort={5}}




%% Fonctions %%%%%%%%%%%%%%%%%%%%%%%%%%%%%%%%%%%%%%%%%%%%%%%%%%%%%%%%%%

\newglossaryentry{f-signe}{type=functions, 
    name={sgn},
    description={Fonction signe},
    sort={1}}

\newglossaryentry{f-ind}{type=functions, 
    name={\ensuremath{\iota_{\mathcal{A}}}},
    description={Fonction indicatrice sur l'ensemble $\mathcal{A}$},
    sort={2}}


%% Probabilité %%%%%%%%%%%%%%%%%%%%%%%%%%%%%%%%%%%%%%%%%%%%%%%%%%%%%%%%%%

\newglossaryentry{proba-N}{type=probas, 
    name={\ensuremath{\mathcal{N}(\mu, \sigma^2)}},  % \parbox{2.8cm}{\ensuremath{x\sim \mathcal{N}(m, \sigma^2)}}
    description={Loi normale de moyenne $\mu$ et d'ecart-type $\sigma$},
    sort={1}}

\newglossaryentry{proba-den}{type=probas, 
    name={\ensuremath{P(x)}},
    description={Densité de probabilité de $x$},
    sort={2}}
%\multicolumn{2}{c}{  \sim \mathcal{N}(m, \sigma^2)


%% Dimensions %%%%%%%%%%%%%%%%%%%%%%%%%%%%%%%%%%%%%%%%%%%%%%%%%%%%%%%%%%

\newglossaryentry{P}{
    type=dimensions,
    name={\ensuremath{P}},
    description={Nombre de pixels},
    sort={01}
}

\newglossaryentry{M}{
    type=dimensions,
    name={\ensuremath{M}},
    description={Nombre de canaux},
    sort={02}
}

\newglossaryentry{N}{
    type=dimensions,
    name={\ensuremath{N}},
    description={Nombre de pixels acquis},
    sort={03}
}

\newglossaryentry{Rt}{
    type=dimensions,
    name={\ensuremath{R_{\mathrm{true}}}},
    description={Dimension du véritable sous-espace signal},
    sort={03}
}

\newglossaryentry{R}{
    type=dimensions,
    name={\ensuremath{R}},
    description={Dimension estimée du véritable sous-espace signal},
    sort={04}
}


%% Modèle direct %%%%%%%%%%%%%%%%%%%%%%%%%%%%%%%%%%%%%%%%%%%%%%%%%%%%%%%

\newglossaryentry{r}{
    type=modele,
    name={\ensuremath{r}},
    description={Rapport d'acquisition $\gls{N}/\gls{P}$},
    sort={01}
}

\newglossaryentry{I}{
    type=modele,
    name={\ensuremath{\mathcal{I}}},
    description={Ensemble des index des positions spatiales acquises},
    shape={\gls{N}},
    sort={02}
}

\newglossaryentry{Phi}{
    type=modele,
    name={\ensuremath{\boldsymbol\Phi}},
    description={Opérateur de sous-échantillonnage spatial tel que $\mathbf{Y}_{\mathcal{I}} = \mathbf{Y}\Phi$},
    shape={\gls{P}\ensuremath{\times}\gls{N}},
    sort={03}
}

\newglossaryentry{Y}{
    type=modele,
    name={\ensuremath{\mathbf{Y}}},
    description={Matrice qui correspondrait aux données EELS complètes},
    shape={\gls{M} \ensuremath{\times} \gls{P}},
    sort={04}
}

\newglossaryentry{Yi}{
    type=modele,
    name={\ensuremath{\mathbf{Y}_\mathcal{I}}},
    description={Matrice d'observation},
    shape={\gls{M} \ensuremath{\times} \gls{N}},
    sort={05}
}

\newglossaryentry{X}{
    type=modele,
    name={\ensuremath{\mathbf{X}}},
    description={Données inconnues à reconstruire},
    shape={\gls{M} \ensuremath{\times} \gls{P}},
    sort={06}
}

\newglossaryentry{E}{
    type=modele,
    name={\ensuremath{\mathbf{E}}},
    description={Matrice de bruit},
    shape={\gls{M} \ensuremath{\times} \gls{N}},
    sort={07}
}

\newglossaryentry{sig}{
    type=modele,
    name={\ensuremath{\sigma}},
    description={\'Ecart-type du bruit blanc additif gaussien},
    sort={08}
}

\newglossaryentry{Xh}{
    type=modele,
    name={\ensuremath{\hat{\mathbf{X}}}},
    description={L'image reconstruite},
    shape={\gls{M} \ensuremath{\times} \gls{P}},
    sort={09}
}


%% Démélange %%%%%%%%%%%%%%%%%%%%%%%%%%%%%%%%%%%%%%%%%%%%%%%%%%%%%%%%%%%
\newglossaryentry{Mu}{
    type=demelange,
    name={\ensuremath{\mathbf{M}}},
    description={Matrice des composantes spectrales},
    shape={\gls{M} \ensuremath{\times} \gls{P}},
    sort={1}
}

\newglossaryentry{A}{
    type=demelange,
    name={\ensuremath{\mathbf{A}}},
    description={Matrice des abondances},
    shape={\gls{M} \ensuremath{\times} \gls{P}},
    sort={2}
}

%% ACP %%%%%%%%%%%%%%%%%%%%%%%%%%%%%%%%%%%%%%%%%%%%%%%%%%%%%%%%%%%%%%%%%

\newglossaryentry{H}{
    type=acp,
    name={\ensuremath{\mathbf{H}}},
    description={Base des composantes principales associées aux données},
    shape={\gls{M} \ensuremath{\times} \gls{M}},
    sort={1}
}

\newglossaryentry{d2}{
    type=acp,
    name={\ensuremath{(d_b^2)_b}},
    description={Valeurs propres associées aux colonnes de \gls{H}},
    shape={\gls{M}},
    sort={2}
}

\newglossaryentry{d}{
    type=acp,
    name={\ensuremath{d^2}},
    description={Valeurs propres associées aux colonnes de \gls{H}},
    shape={\gls{M}},
    sort={2}
}

\newglossaryentry{S}{
    type=acp,
    name={\ensuremath{\mathbf{S}}},
    description={Coefficients de représentation des données dans la base \gls{H}},
    shape={\gls{M} \ensuremath{\times} \gls{P}},
    sort={3}
}



% %% 3S %%%%%%%%%%%%%%%%%%%%%%%%%%%%%%%%%%%%%%%%%%%%%%%%%%%%%%%%%%%%%%%%%%

%% Operators
\newglossaryentry{D}{
    type=chap3,
    name={\ensuremath{\mathbf{D}}},
    description={Opérateur de gradient spatial discret avec $P' = m(n-1) + (m-1)n$, où $m$ et $n$ sont respectivement le nombre de lignes et de colonnes de l'image},
    shape={\gls{P} \ensuremath{\times} \gls{P}'},
    sort={01}
}

\newglossaryentry{Delta}{
    type=chap3,
    name={\ensuremath{\Delta}},
    description={Opérateur de laplacien spatial discret, $\Delta = -\mathbf{D}\mathbf{D}^T$},
    shape={\gls{P} \ensuremath{\times} \gls{P}},
    sort={02}
}

% Hyperparamètres
\newglossaryentry{ls2n}{
    type=chap3,
    name={\ensuremath{\lambda_{\mathrm{S2N}}}},
    description={Premier paramètre de la méthode S2N},
    sort={03}
}

\newglossaryentry{ms2n}{
    type=chap3,
    name={\ensuremath{\mu_{\mathrm{S2N}}}},
    description={Second paramètre de la méthode S2N},
    sort={04}
}

\newglossaryentry{m3s}{
    type=chap3,
    name={\ensuremath{\mu_{\mathrm{3S}}}},
    description={Paramètre de la méthode 3S},
    sort={05}
}

\newglossaryentry{w}{
    type=chap3,
    name={\ensuremath{w}},
    description={Poids associés à la méthode 3S},
    sort={06}
}

\newglossaryentry{hsig}{
    type=chap3,
    name={\ensuremath{\hat{\sigma}}},
    description={\'Ecart-type estimé du bruit blanc additif gaussien},
    sort={07}
}


% %% CLS %%%%%%%%%%%%%%%%%%%%%%%%%%%%%%%%%%%%%%%%%%%%%%%%%%%%%%%%%%%%%%%%%

\newglossaryentry{lcls}{
    type=chap4,
    name={\ensuremath{\lambda_{\mathrm{CLS}}}},
    description={Paramètre de la méthode CLS},
    sort={01}
}






%% ==============================================================
%% Subappendix

% Add appendix to each chapter
\usepackage[toc,page]{appendix}
\usepackage{chngcntr}


%% ==============================================================
%% Style de page

\usepackage{fancyhdr}

% Mark style to make trucature.
\usepackage[fit]{truncate}
\newcommand{\markformat}[1]{\truncate{0.95\textwidth}{\footnotesize\scshape\nouppercase{#1}}}

% Fancyhdr styles
%

\fancypagestyle{fancyfrontmatter}{%
   \fancyhf{}
   %
   \renewcommand\chaptermark[1]{\markboth{##1}{}}
   %
   \fancyhead[LE,RO]{\footnotesize\thepage}%
   \fancyhead[RE,LO]{\markformat{\leftmark}}%
   \renewcommand{\headrulewidth}{1pt}
}

\fancypagestyle{fancymainmatter}{%
    \fancyhf{}
    %
    \renewcommand\chaptermark[1]{\markboth{\chaptername~\thechapter. ##1}{}}
    \renewcommand\sectionmark[1]{\markright{\thesection. ##1}}
    %
    \fancyhead[LO]{\markformat{\rightmark}}
    \fancyhead[RE]{\markformat{\leftmark}}
    \fancyhead[LE, RO]{\footnotesize\thepage}
}

% The plain style is used for chapters first page.
\fancypagestyle{plain}{ %
    \fancyhf{} % remove everything
    \renewcommand{\headrulewidth}{0pt} % remove lines as well
    \renewcommand{\footrulewidth}{0pt}
}


% Redefine matter commands
%
\renewcommand\frontmatter{%
    \cleardoublepage%
    \pagenumbering{roman}%  Page Number with i, ii, iii, iv, ...
    \pagestyle{fancyfrontmatter}
}

\renewcommand\mainmatter{%
    \cleardoublepage%
    \pagenumbering{arabic}%  Page Number with 1, 2, ...
    \pagestyle{fancymainmatter}
}

\renewcommand\backmatter{%
    \cleardoublepage%
    \pagestyle{empty}
}

% Disable glossaries mark
\renewcommand{\glsglossarymark}[1]{}


%% ==============================================================
%% Text packages

\usepackage{xcolor}  % Colors
\usepackage{enumerate}
\usepackage[shortlabels]{enumitem}       % personnalisation des enumerate
\setlist[itemize]{label=$\square$}  % black

%% ==============================================================
%% Graphics packages

% Include graphic
\usepackage{graphicx}
\DeclareGraphicsExtensions{.pdf,.jpg,.png}

% Subfigures
\usepackage[center]{subfigure}

% Tikz config
\usepackage{style/tikzstyle}

%\makeatletter
%\setlength{\@tufte@caption@vertical@offset}{\baselineskip}
%\makeatother


%% ==============================================================
%% Algorithms

% This is required to make [H] for algorithms
% so that no error appears when inside a minipage.
\usepackage{float}

\usepackage[
    french,     % French language.
    tworuled,   % Two rules : one at top and one at bottom.
    vlined,      % For vertical lines in loops.
    linesnumbered
    ]{algorithm2e}

\NoCaptionOfAlgo      % No caption.
\DontPrintSemicolon   % No semicolumn at the line end.

% Solution to create new tufte-like algorithme environments.
% cf. https://tex.stackexchange.com/questions/113631/caption-placement-for-new-float-in-tufte-book-class
\newcounter{algorithme}[chapter]
\newcommand\algorithmename{Algorithme}
\newcommand\listalgorithmename{Liste d'algorithmes}
\makeatletter
\newcommand\listofalgorithmes{%
    \ifthenelse{\equal{\@tufte@class}{book}}%
    {\chapter*{\listalgorithmename}}%
    {\section*{\listalgorithmename}}%
    %  \begin{fullwidth}%
    \@starttoc{loa}%
    %  \end{fullwidth}%
}
\renewcommand\thealgorithme{\ifnum \c@chapter>\z@ \thechapter.\fi \@arabic\c@algorithme}
\def\fps@algorithme{tbp}
\def\ftype@algorithme{1}
\def\ext@algorithme{loe}
\def\fnum@algorithme{\algorithmename\nobreakspace\thealgorithme}
\newenvironment{algorithme}[1][htbp]
    {\begin{@tufte@float}[#1]{algorithme}{}}
    {\end{@tufte@float}}
\let\l@algorithme\l@figure
\makeatother


%% ==============================================================
%% Tabular packages

\usepackage{tabularx}
\usepackage{multirow, bigdelim}
\usepackage{booktabs}     % For serious tables (\toprule, \midrule, \bottomrule).

\usepackage{array}
\newcolumntype{M}[1]{>{\centering\arraybackslash}m{#1}}


%% ==============================================================
%% FontAwesome

% Fontawesome icons
\usepackage{fontawesome}
\usepackage{colorbrewer}

% Loops in tikz-style
\usepackage{pgffor}

% Mark and check commands.
\newcommand{\minusfa}[1][1]{%
    \foreach \n in {1,...,#1}{\color{Set1-3-1}\faicon{minus-square}}%
}
\newcommand{\plusfa}[1][1]{%
    \foreach \n in {1,...,#1}{\color{Set1-3-3}\faicon{plus-square}}%  plus plus-square plus-circle
}
\newcommand{\checkfa}{{\color{black}\faicon{check-square}}}  % check check-square check-circle


%% ==============================================================
%% Landscape


\usepackage{pdflscape}
\usepackage{afterpage}
\usepackage{environ}

\def\mylandscapebody#1{%
    \afterpage{%%
        %
        \thispagestyle{empty}
        %
        \begin{landscape}
            \newgeometry{
                asymmetric,
                a4paper,
                left=24.8mm, top=27.4mm,
                headsep=2\baselineskip,
                textwidth=49\baselineskip, % 107mm + 49.4mm
                textheight=164.6mm, % 49\baselineskip (241.1mm) - 49.4mm = 183.5m
                marginparsep=8.2mm, marginparwidth=0mm, %49.4mm,
                headheight=\baselineskip
            }
            #1
            % \lipsum[1-10]
            \restoregeometry%
        \end{landscape}
    }%
}

\NewEnviron{mylandscape}{%%
    \expandafter\mylandscapebody\expandafter{\BODY}%%
}


%% ==============================================================
%% Bibtex

% To perform the publication list
% \usepackage{bibunits}

% % To remove the header for bibunits
% \usepackage{etoolbox}
% \AtBeginEnvironment{bibunit}{\renewcommand\chapter[5]{}}

% \newcommand{\localbib}[1]{
%     \begin{bibunit}[bib/StyleThese]
%         \nocite{*}
%         \putbib[#1]
%     \end{bibunit}
% }

%% ==============================================================
%% Misc

\newcommand{\twonorm}[1]{\ensuremath{||#1||_2^2}}
\newcommand{\frobnorm}[1]{\ensuremath{||#1||_{\mathrm{F}}^2}}

\newcommand{\cf}{\textit{cf.}}
\newcommand{\ie}{{c.-à-d.}}
\newcommand{\eg}{{p. ex.}}
\newcommand{\etal}{et co-auteurs}  %\textit{et al.}}

\newcommand{\tabname}{table}

\newcommand{\num}{n\textsuperscript{o}}
\newcommand{\nums}{n\textsuperscript{os}}

\newcommand{\guillemets}[1]{\og #1\fg{}}

\newcommand{\taille}[2]{\ensuremath{\gls{#1}\times\gls{#2}}}

\newcommand{\argmin}{\ensuremath{\operatornamewithlimits{argmin}}}



\newlength{\tmplength}

\usepackage{lipsum}
\usepackage{comment}

% \usepackage{cite}

\usepackage{hyperref}
\hypersetup{
    linktoc=all,
    breaklinks=true,
    colorlinks=true,
    linkcolor=mydarkblue,
    citecolor=mydarkblue,
    filecolor=mydarkblue,
    urlcolor=mydarkblue
}

%% ==============================================================
%% Pretty refs and Todo

\usepackage[noabbrev]{cleveref}

\usepackage{bookmark} % https://tex.stackexchange.com/questions/164248/defining-a-chapter-outside-the-last-part-amsbook
\usepackage[nottoc]{tocbibind}  % Bib in toc


