%!TEX root = ../main.tex

\begin{fullwidth}
\chapter*{Remerciements} % (fold)
\label{ch:remerciements}
\addcontentsline{toc}{chapter}{Remerciements}

    Au terme de ce travail de thèse, j'ai à l'esprit de nombreuses personnes m'ayant permis de m'accomplir et de réaliser le travail synthétisé par ce manuscrit.
    
    Je souhaiterais d'abord remercier mes deux directeurs de thèse Nicolas et Thomas, ainsi que Nathalie avec qui nous avons collaboré. Ils m'ont montré leur confiance en m'acceptant comme doctorant et j'ai apprécié leur patience, leur disponibilité, leurs conseils et leurs idées. Cela a réellement été un grand plaisir de travailler avec eux.
    
    Je remercie ensuite très chaleureusement les membres du jury pour leur investissement. Merci à Cécile Hébert et à Vincent Mazet d'avoir accepté de rapporter cette thèse, et à Jérôme Idier, Férréol Soulez et Cyril Ruckebush d'avoir examiné mon travail. Cela a été un grand honneur de vous présenter mes travaux.
    
    Ensuite, je souhaiterais remercier tout particulièrement les membres de l'équipe \textsc{SC} pour cette ambiance très positive et accueillante. Vous avez tous participé à ce savant mélange qui rend une thèse très agréable et j'espère y avoir apporté une touche de folie, d'excentricité et de gaffes (Claire, c'est pour toi). 
    %
    Je remercie tous particulièrement Adrien, Louis et Olivier qui ont débuté leur thèse en même temps que moi et avec qui j'ai eu la joie de partager beaucoup d'interrogations et de rires. Merci à tous les doctorants passés, Pierre-Antoine, Yanna et Vinicius F., qui ont été des exemples pour le jeune doctorant que j'ai été. Merci aux doctorants arrivés par la suite, Maxime, Claire, Camille, Pierre-Hugo, Asma, Vinicius O. que j'ai eu la joie d'accueillir de mon mieux et avec qui j'ai pu également passer des moments de qualité. 
    %
    Merci à Baha, mon collègue d'ATER, avec qui j'ai partagé beaucoup de discussions passionnantes, de sourires communicatifs, de cours et de corrections.
    %
    Merci à tous les autres doctorants et post-doctorants avec qui j'ai pu passer ces années de thèse~: Alberto, Dylan, Tatsumi, Mouna, Sixin, Paul et Dana.
    %
    Merci en particulier à Charly qui a été mon premier contact au sein de l'équipe, à Marie, Nathalie, Cédric et Emmanuel qui m'ont également soutenu et accompagné par leurs conseils et leur écoute.
    
    Je remercie également Annabelle avec qui j'ai pu partager des nouvelles de ma petite famille, ainsi que toutes les secrétaires, SAM, Muriel et Isabelle, qui m'ont aidées dans les tâches administratives. J'ai pu apprécier leur bonne humeur, leurs attentions et leur soutien au cours de ces années de thèse. Merci plus largement à Agnès Requis et Marie-Claude Portell qui m'ont beaucoup aidé dans mes démarches auprès de l'école doctorale et de l'INPT. 
    
    Merci également à l'ensemble du groupe STEM du LPS qui m'ont accueillis lors de mes passages. Merci plus particulièrement à Odile Stéphan qui a soutenu l'AAP Imag'In et l'ensemble du projet. Merci à Marcel Tencé, Alberto Zobelli et Anna Tararan qui ont initié le projet \guillemets{random scanning}, Marta de Frutos et Xiaoyan Li qui ont réalisé les acquisitions et fournit les échantillons et à Alexandre Gloter qui a fourni des échantillons.
    
    Je migre doucement vers les remerciements moins professionnels en remerciant de nombreux amis qui me soutiennent et me font grandir depuis longtemps. Je pense tout particulièrement à Bertrand et à sa famille qui m'offrent leur amitié depuis mes 8 ans et à Baptiste dont l'amitié, bien que plus récente, est tout aussi riche. Merci à ces amis qui, comme moi, n'ont pas attendu la stabilité pour vivre une vie de famille heureuse et avec qui j'ai pu partager mes difficultés et mes joies~: Sandra et Benjamin, Bénédicte et Xavier-Marie, Anne-Marie et Augustin, Maïa et \'Edouard. Merci à Sr. Clara de m'accompagner de ses prières depuis son monastère. Merci vous tous, autres amis, éloignés par la distance pour la plupart mais toujours présents avec joie.
    
    Merci à ma famille, pour cet amour et ce soutien infaillible qui ont fait de moi ce que je suis. Merci à ma belle-famille, qui m'apportent un grand soutien et une grande aide. Merci à ma merveilleuse moitié, Magdalena, qui m'accompagne tous les jours et à mes deux enfants, Louise et Théophile.
    
    Enfin, je termine ces remerciements par un professeur, Jean-Michel Ferrard, qui m'a enseigné les mathématiques à mon arrivée en classes préparatoires. Il m'a donné le goût des mathématiques et de la rigueur (et de \LaTeX). Je le remercie pour avoir répondu patiemment à toutes mes questions de l'époque et d'avoir initié, avec beaucoup d'autres, la trajectoire conduisant à ce manuscrit. 
    
    \newpage\thispagestyle{empty}
    
\end{fullwidth}

% chapter remerciements (end)
