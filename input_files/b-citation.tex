%!TEX root = ../main.tex

\cleardoublepage
\thispagestyle{empty}
\begin{fullwidth}
%
~\vspace{10\baselineskip}

\hfill
\begin{minipage}{10cm}
    {\nohyphenation\noindent%\fontsize{14}{18}\selectfont
        Mais\dots chanter,\\
        Rêver, rire, passer, être seul, être libre,\\
        Avoir l'oeil qui regarde bien, la voix qui vibre,\\
        Mettre, quand il vous plaît, son feutre de travers,\\
        Pour un oui, pour un non, se battre, - ou faire un vers !\\
        Travailler sans souci de gloire ou de fortune,\\
        À tel voyage, auquel on pense, dans la lune !\\
        N'écrire jamais rien qui de soi ne sortît,\\
        Et modeste d'ailleurs, se dire : mon petit,\\
        Sois satisfait des fleurs, des fruits, même des feuilles,\\
        Si c'est dans ton jardin à toi que tu les cueilles !\\[10pt]
%        \dots\\
%        Bref, dédaignant d’être le lierre parasite,\\
%        Lors même qu’on n’est pas le chêne ou le tilleul,\\
%        Ne pas monter bien haut, peut-être, mais tout seul !\\[10pt]
        \begin{flushright}
        {\large\color{myblue}%
            Edmond Rostand\\
            {\itshape Cyrano de Bergerac, Acte II, scène 8}.
        }
        \end{flushright}
    
        \vspace{1.5cm}
        
        - [\ldots] Qu'est-ce que signifie ``apprivoiser'' ?\\
        - C'est une chose trop oubliée, dit le renard. \c Ca signifie ``créer des liens\dots''.\\[10pt]
        \begin{flushright}
            {\large\color{myblue}%
                Antoine de Saint-Exupéry\\
                {\itshape Le petit prince, Chapitre XXI}.
            }
        \end{flushright}
    }
\end{minipage}


\end{fullwidth}