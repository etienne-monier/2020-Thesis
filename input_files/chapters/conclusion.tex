\chapter{Conclusion}\label{chap-conclusion}

Dans cette thèse, nous avons étudié la possibilité de limiter la détérioration des échantillons sensibles en spectromicroscopie EELS par acquisition partielle de l'image multi-bande, couplée à une reconstruction a posteriori des pixels manquants. 
%
Plus précisément, les images correspondent à des acquisitions incomplètes en balayage aléatoire, ce qui permet le découplage des informations spatiales et spectrales.
%
Cette technique détaillée au chapitre~\ref{ch-chapter_1} a été rendue possible par le développement de systèmes d'acquisition hautement paramétrables~\cite{tararan2016random, zobelli2019spatial, tence2019following} et cela permet à la fois d'étudier l'évolution temporelle de la zone analysée et de limiter les dégâts d'irradiation dus à l'accumulation de la dose électronique.
%
Or, les expérimentateurs recherchent de plus en plus à traiter les données en ligne, \ie{} au cours de l'acquisition, ce qui requiert une technique de reconstruction d'autant plus rapide que l'acquisition est de courte durée. Toutefois, l'état de l'art réalisé dans le chapitre~\ref{ch-chapter_2} met en lumière une lacune puisque les méthodes utilisées dans la littérature sont soit rapides mais peu performantes, soit très efficaces mais lentes. C'est cette lacune que ce travail de thèse a tenté de combler en proposant des solutions rapides et performantes à la fois.

Une approche prometteuse pour respecter les contraintes de rapidité et de performances réside dans les approches par \glsentrylong{mc} régularisés, plus rapides que les méthodes par \glsentrylong{ad} et plus performantes que les techniques par \glsentrylong{ppv}. Dans le chapitre~\ref{ch-chapter_3}, deux méthodes ont été proposées dans le cadre d'images spatialement lisses, reposant sur des régularisations spectrales et spatiales adaptées. 
%
En particulier, la nature faible-rand de l'image reconstruite a été contrainte par une norme nucléaire pour la méthode S2N et par une \glsentrylong{acp} dont seules les composantes les plus puissantes ont été conservées pour 3S.
%
Des expériences conduites sur des données synthétiques et réelles ont mis en évidence l'intérêt de l'échantillonnage partiel pour restituer une information spectrale riche, tandis que, à dose totale d'électron équivalente, les approches par échantillonnage complet semblaient rendre un meilleur contenu spatial.

Dans le chapitre~\ref{ch-chapter_4}, une autre méthode par \glsentrylong{mc} régularisés appelée CLS a été proposée dans le cadre d'échantillons spatialement périodiques. En particulier, la régularisation $\ell_{2,1}$ a été utilisé pour favoriser les reconstructions présentant une grande parcimonie groupée dans la base \gls{dct}. Des expériences menées sur des images synthétiques et réelles ont comparés les performances de CLS et des techniques rencontrées en reconstruction d'images en microscopie. Les résultats ont montré l'intérêt de CLS comme compromis de rapidité et de précision.


\section*{Discussions et perspectives}

\paragraph{Intérêt de l'inpainting par rapport au débruitage.} Dans la section~\ref{sec-lr-demelange-res}, nous avons tenté d'évaluer le gain en performances de la technique d'échantillonnage partiel couplé à la reconstruction par rapport à une acquisition complète couplée à une étape de débruitage, la dose d'électron totale utilisée au cours de l'acquisition restant constante. Les résultats ont permis de conclure que l'image obtenue après débruitage présente un contenu spatial plus riche tandis que l'image obtenue par reconstruction affiche une meilleure qualité spectrale.

D'autres études similaires sont parfois parvenues à des conclusions différentes, comme~\cite{trampert2018ultramicroscopy} qui a comparé entre autres les deux approches et qui a déterminé que l'approche par inpainting était supérieure. Au contraire,~\cite{sanders2020inpainting} a conclu que l'acquisition conventionnelle couplée à une étape de débruitage donne de meilleurs résultats si le bruit est de nature poissonienne, tandis que les performances sont semblables pour un bruit mixte. En définitive, pour une situation donnée, il est compliqué de savoir quelle approche privilégier.

\paragraph{Intérêt de la super-résolution en basse résolution.} La section~\ref{sec-donnees-sous-echantillonnees} a décrit la nécessité pour l'expérimentateur de réduire volontairement la résolution de l'image acquise en vue de préserver l'échantillon. Dès lors, une solution possible pour restituer la résolution permise par le pouvoir séparateur de l'instrument est la super-résolution. D'un point de vue plus général, un sous-échantillonnage régulier a été proposé dans~\cite{trampert2018ultramicroscopy} comme alternative au sous-échantillonnage aléatoire pour des images 2D SEM. Cependant, les expériences conduites indiquent que le masque régulier conduit à des performances moindres. Puisque cette approche fait appel à des algorithmes particuliers de super-résolution, il s'agit là d'une alternative supplémentaire aux méthodes d'inpainting et de débruitage comparées ci-dessus qui pourrait être approfondie pour des images multi-bande. 

\paragraph{Intérêt de l'imagerie HAADF pour la reconstruction de l'image EELS.} Une source d'information forte non-utilisée dans ce manuscrit est l'imagerie HAADF plus rapide que l'imagerie EELS et moins destructrice pour l'échantillon. En effet, une acquisition HAADF de même résolution et de même taille que l'image EELS à acquérir peut être réalisée avant l'acquisition EELS. Dès lors, l'on peut en déduire beaucoup d'information concernant la structure spatiale de l'échantillon et s'en servir à la fois pour la reconstruction et pour le choix du chemin d'acquisition (\cf\ paragraphe suivant). Il s'agit d'une perspective importante puisque cela permettrait à la fois de pondérer la régularisation de Sobolev dans 3S et S2N, mais également de fixer le support de la \gls{dct} dans CLS. Remarquons toutefois que certaines structures visibles dans l'image EELS peuvent ne pas apparaître dans l'image HAADF.

\paragraph{Meilleur choix du chemin d'acquisition.} Au cours des expériences menées dans les chapitres~\ref{ch-chapter_3} et~\ref{ch-chapter_4}, le chemin d'acquisition a été aléatoirement et uniformément choisi. Puisque les performances dépendent également fortement du masque d'échantillonnage, de nombreux autres masques statiques et de méthodes d'échantillonnage dynamique ont été développés, \cf\ section~\ref{sec-art-micro}. En particulier, deux méthodes peuvent être réalisées assez facilement.

La première consiste à utiliser l'imagerie STEM-HAADF comme décrit dans le paragraphe précédent. La structure spatiale de l'échantillon peut être déduite de l'image 2D et un chemin intelligent visitera davantage les contours des structures pour permettre une meilleure reconstruction. Cependant, une part aléatoire devra également être conservée dans le choix du chemin d'acquisition pour visiter les zones lisses car l'image HAADF peut ne pas révéler toutes les structures de l'échantillon. L'imagerie HAADF peut ainsi contribuer à améliorer la qualité de la reconstruction en réalisant un meilleur choix de chemin d'acquisition.

La seconde, assez similaire, consiste à réaliser une première acquisition EELS standard avec une dose d'électron très faible et à localiser les contours des structures. Une seconde acquisition à dose totale plus élevée peut ensuite être effectuée sur les contours détectés. Enfin, une étape de fusion restitue une image reconstruite. Une méthode semblable a été réalisée dans~\cite{dahmen2016feature} pour l'imagerie SEM. 


\paragraph{\`A propos du modèle de bruit.} La section~\ref{sec-nature-bruit} a décrit la nature mixte poisson-gaussien du bruit d'acquisition. Toutefois, pour plus de simplicité, les méthodes et les expériences ont été élaborées en ne considérant qu'un bruit gaussien. Ceci est discutable, d'autant plus que des travaux~\cite{sanders2020inpainting} ont mis en évidence l'influence du modèle de bruit sur la qualité de reconstruction. Pour s'assurer de la validité des méthodes proposées, des expériences ont été conduites à l'annexe~\ref{sec-bruit-mixte} pour évaluer la robustesse de CLS dans le cas d'une image entachée d'un bruit mixte. Les résultats ont permis de conclure que l'ajout d'une composante poissonienne n'influençais pas les performances de manière significative. Toutefois, une perspective intéressante concerne l'adaptation des méthodes par \glsentrylong{mc} régularisés dans le cas d'un bruit mixte où des techniques de stabilisation de variance comme la transformée de Anscombe~\cite{anscombe1948transformation} pourraient être utilisées. Malheureusement, ces techniques nécessitent la connaissance de la statistique du bruit d'acquisition, ce qui reste compliqué.


\paragraph{Méthodes par réseaux de neurones.} Le travail exposé dans ce manuscrit s'est focalisé sur la reconstruction \emph{rapide} et aucune méthode \emph{coûteuse} mais performante n'a été développé. Toutefois, des techniques par réseaux de neurones convolutifs ont été décris à la section~\ref{sec-methodes-convnets} et il serait intéressant de les comparer à l'algorithme populaire BPFA.


% sanders2020inpainting



% - Reconstruction en ligne : comment mettre à jour l'espace dans lequel vivent les données ? (GROUSE https://arxiv.org/abs/1006.4046, descente de riemanienne)

% - Correction du drift
