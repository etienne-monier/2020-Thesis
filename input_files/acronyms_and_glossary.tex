%!TEX root = ../main.tex
% To build the glossary: makeglossaries main

%%%%% Acronyms
% \newacronym[plural={<plural acronym>},
%             first={<text displayed at first occurrence>},
%             firstplural={<idem, with plural>}]
%             {<label>}
%             {<acronym>}
%             {<full name to display in acronym section>}

% Laboratoires
\newacronym{irit}
           {IRIT}
           {Institut de Recherche en Informatique de Toulouse}

\newacronym{lps}
           {LPS}
           {Laboratoire de Physique des Solides}

% Microscopes
\newacronym{stem}
           {STEM}
           {Scanning Transmission Electron Microscope}

\newacronym{tem}
           {TEM}
           {Transmission Electron Microscope}

\newacronym{sem}
           {SEM}
           {Scanning Electron Microscope}

\newacronym{mfa}
           {MFA}
           {microscopie à force atomique}

% Modalités
\newacronym{haadf}
           {HAADF}
           {High-Angle Annular Dark-Field}

\newacronym{eels}
           {EELS}
           {Electron Energy Loss Spectroscopy}

\newacronym{edx}
           {EDX}
           {analyse dispersive en énergie}


% Techniques
\newacronym{pca}
           {ACP}
           {Analyse par Composantes Principales}

\newacronym{cs}
           {CS}
           {Compressed Sensing}

\newacronym{ppv}
           {PPV}
           {Plus Proches Voisins}

\newacronym{ad}
           {AD}
           {Apprentissage de Dictionnaire}

\newacronym{ebi}
           {EBI}
           {Exemplar-Based Inpainting}

\newacronym{mc}
           {MC}
           {moindre carré}
        
\newacronym{dct}
           {DCT}
           {transformée en cosinus discrète}

\newacronym{tv}
           {TV}
           {variation totale}

%%%% Glossary entries


%% Notations génériques

\newglossaryentry{g-a}{type=notgen, name={\ensuremath{a}}, description={Scalaire}, sort={01}}

\newglossaryentry{g-av}{type=notgen,
	name={\ensuremath{\mathbf{a}}},
	description={Vecteur colonne}, 
	sort={02}}

\newglossaryentry{gavi}{type=notgen, 
	name={\ensuremath{\mathbf{a}_i}}, 
	description={$i^\text{ème}$ composante du vecteur \gls{g-av}}, 
	sort={03}}

\newglossaryentry{g-A}{type=notgen, 
	name={\ensuremath{\mathbf{A}}},
	description={Matrice},
	sort={04}}

\newglossaryentry{g-Aij}{type=notgen,
	name={\ensuremath{\mathbf{A}_{ij}}},
	description={Coefficient $(i, j)$ de la matrice \gls{g-A}},
	sort={05}}

\newglossaryentry{g-Aj}{type=notgen, 
	name={\ensuremath{\mathbf{A}_j}},
	description={$j^\text{ème}$ colonne de la matrice \gls{g-A}},
	sort={06}}

\newglossaryentry{g-Ai}{type=notgen, 
	name={\ensuremath{\mathbf{A}_{i, :}}},
	description={$i^\text{ème}$ ligne de la matrice \gls{g-A}},
	sort={07}}

\newglossaryentry{g-T}{type=notgen, 
	name={\ensuremath{(\cdot)^{T}}},
	description={Transposée},
	sort={08}}

\newglossaryentry{g-pm}{type=notgen, 
	name={\ensuremath{\mathbf{AB}}},
	description={Produit matriciel},
	sort={09}}

\newglossaryentry{g-nf}{type=notgen, 
	name={\ensuremath{||\mathbf{A}||_F}},
	description={Norme de Frobenius de \gls{g-A}},
	sort={10}}



%% Dimensions
\newglossaryentry{P}{
	type=notation,
	name={\ensuremath{P}},
	description={Le nombre de pixels},
	sort={01}
}

\newglossaryentry{M}{
	type=notation,
	name={\ensuremath{M}},
	description={Le nombre de cannaux},
	sort={02}
}

\newglossaryentry{N}{
	type=notation,
	name={\ensuremath{N}},
	description={Le nombre de pixels acquis},
	sort={03}
}

\newglossaryentry{r}{
	type=notation,
	name={\ensuremath{r}},
	description={Le rapport d'acquisition $\gls{N}/\gls{P}$},
	sort={04}
}

%% Variance
\newglossaryentry{sig}{
	type=notation,
	name={\ensuremath{\sigma}},
	description={La variance du bruit blanc additif gaussien},
	sort={05}
}

%% Sets
\newglossaryentry{I}{
	type=notation,
	name={\ensuremath{\mathcal{I}}},
	description={L'ensemble des index des positions spatiales acquises},
	shape={(\gls{N}, )},
	sort={06}
}

%% Matrices
\newglossaryentry{X}{
	type=notation,
	name={\ensuremath{\mathbf{X}}},
	description={Les données inconnues à restituer},
	shape={(\gls{M}, \gls{P})},
	sort={07}
}

\newglossaryentry{Y}{
	type=notation,
	name={\ensuremath{\mathbf{Y}}},
	description={La matrice d'observation},
	shape={(\gls{M}, \gls{N})},
	sort={08}
}

\newglossaryentry{B}{
	type=notation,
	name={\ensuremath{\mathbf{B}}},
	description={La matrice de bruit gaussien},
	shape={(\gls{M}, \gls{N})},
	sort={09}
}

\newglossaryentry{Xh}{
    type=notation,
    name={\ensuremath{\hat{\mathbf{X}}}},
    description={Les données reconstruites},
    shape={(\gls{M}, \gls{P})},
    sort={10}
}

