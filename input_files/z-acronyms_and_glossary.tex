%!TEX root = ../main.tex
% To build the glossary: makeglossaries main

%%%%% Acronyms
% \newacronym[plural={<plural acronym>},
%             first={<text displayed at first occurrence>},
%             firstplural={<idem, with plural>}]
%             {<label>}
%             {<acronym>}
%             {<full name to display in acronym section>}

% Laboratoires
\newacronym{irit}{IRIT}{institut de recherche en informatique de Toulouse}

\newacronym{lps}{LPS}{laboratoire de physique des solides}

% Microscopes
\newacronym{stem}{STEM}{scanning transmission electron microscope}

\newacronym{tem}{TEM}{transmission electron microscopy}

\newacronym{sem}{SEM}{scanning electron microscope}

\newacronym{mfa}{MFA}{microscopie à force atomique}

\newacronym{irm}{IRM}{imagerie par résonance magnétique}

% Modalités
\newacronym{haadf}{HAADF}{high-angle annular dark-field}

\newacronym{adf}{ADF}{annular dark-field}

\newacronym{eels}{EELS}{electron energy loss spectroscopy}

\newacronym{edx}{EDX}{spectroscopie de rayons X à dispersion d’énergie}


% Techniques
\newacronym{acp}{ACP}{analyse en composantes principales}

\newacronym{aci}{ACI}{analyse en composantes indépendantes}

\newacronym{cs}{CS}{compressed sensing}

\newacronym{ppv}{PPV}{plus proches voisins}

\newacronym{ad}{AD}{apprentissage de dictionnaire}

\newacronym{ebi}{EBI}{exemplar-based inpainting}

\newacronym{mc}{MC}{moindre carré}
        
\newacronym{dct}{DCT}{transformée en cosinus discrète}

\newacronym{tv}{TV}{variation totale}

\newacronym{goal}{GOAL}{geometric analysis operator learning}

\newacronym{vca}{VCA}{vertex component analysis}

\newacronym{sisal}{SISAL}{simplex identification via variable
    splitting and augmented lagrangian}

\newacronym{sunsal}{SUNSAL}{spectral unmixing by splitting and augmented Lagrangian}

\newacronym{map}{MAP}{maximum a posteriori}

\newacronym{mcmc}{MCMC}{Monte Carlo par chaîne de Markov}


% \newacronym{}{}{}

% Métriques

\newacronym{snr}{SNR}{rapport signal à bruit}

\newacronym{nmse}{NMSE}{normalized mean square error}

\newacronym{asad}{aSAD}{average spectral angle distance}

\newacronym{ssim}{SSIM}{structural similarity}

% Méthodes proposée

\newacronym{s2n}{S2N}{smoothed nuclear norm}

\newacronym{3s}{3S}{smoothed subspace}

\newacronym{cls}{CLS}{cosine least square}

\newacronym{ista}{ISTA}{iterative shrinkage thresholding}

\newacronym{fista}{FISTA}{fast iterative shrinkage thresholding}

\newacronym{bpfa}{BPFA}{beta-process factor analysis}

\newacronym{itkrmm}{ITKrMM}{iterative thresholding and K residual means for masked data}

\newacronym{wksvd}{wKSVD}{weighted K-SVD}







%% Notations génériques %%%%%%%%%%%%%%%%%%%%%%%%%%%%%%%%%%%%%%%%%%%%%%%%

\newglossaryentry{g-a}{
    type=notgen, 
    name={\ensuremath{a}}, 
    description={Scalaire}, 
    sort={01}}

\newglossaryentry{g-av}{type=notgen,
    name={\ensuremath{\mathbf{a}}},
    description={Vecteur colonne}, 
    sort={02}}

\newglossaryentry{gavi}{type=notgen, 
    name={\ensuremath{\mathbf{a}_i}}, 
    description={$i^\text{ème}$ composante du vecteur \gls{g-av}}, 
    sort={03}}

\newglossaryentry{g-A}{type=notgen, 
    name={\ensuremath{\mathbf{A}}},
    description={Matrice},
    sort={04}}

\newglossaryentry{g-Aij}{type=notgen,
    name={\ensuremath{\mathbf{A}_{ij}}},
    description={Coefficient $(i, j)$ de la matrice \gls{g-A}},
    sort={05}}

\newglossaryentry{g-Aj}{type=notgen, 
    name={\ensuremath{\mathbf{A}_j}},
    description={$j^\text{ème}$ colonne de la matrice \gls{g-A}},
    sort={06}}

\newglossaryentry{g-Ai}{type=notgen, 
    name={\ensuremath{\mathbf{A}_{i, :}}},
    description={$i^\text{ème}$ ligne de la matrice \gls{g-A}},
    sort={07}}

\newglossaryentry{g-T}{type=notgen, 
    name={\ensuremath{(\cdot)^{T}}},
    description={Transposée},
    sort={08}}

\newglossaryentry{g-pm}{type=notgen, 
    name={\ensuremath{\mathbf{AB}}},
    description={Produit matriciel},
    sort={09}}

\newglossaryentry{g-pmh}{type=notgen, 
    name={\ensuremath{\mathbf{A\cdot B}}},
    description={Produit matriciel de Hadamard (terme à terme)},
    sort={10}}

\newglossaryentry{g-n1}{type=notgen, 
    name={\ensuremath{||\mathbf{a}||_1}},
    description={Norme $\ell_1$ de \gls{g-av}},
    sort={11}}

\newglossaryentry{g-n2}{type=notgen, 
    name={\ensuremath{||\mathbf{a}||_2}},
    description={Norme $\ell_2$ de \gls{g-av}},
    sort={12}}

\newglossaryentry{g-n12}{type=notgen, 
    name={\ensuremath{||\mathbf{A}||_{2, 1}}},
    description={Norme $\ell_{2, 1}$ de \gls{g-A}},
    sort={13}}

\newglossaryentry{g-nf}{type=notgen, 
    name={\ensuremath{||\mathbf{A}||_\mathrm{F}}},
    description={Norme de Frobenius de \gls{g-A}},
    sort={14}}

\newglossaryentry{g-nn}{type=notgen, 
    name={\ensuremath{||\mathbf{A}||_*}},
    description={Norme nucléaire de \gls{g-A}},
    sort={15}}

\newglossaryentry{g-zvec}{type=notgen, 
    name={\ensuremath{\mathbf{0}_n}},
    description={Vecteur nul de taille $n$},
    sort={16}}

%% Ensembles %%%%%%%%%%%%%%%%%%%%%%%%%%%%%%%%%%%%%%%%%%%%%%%%%%%%%%%%%%

%\newglossaryentry{e-N}{type=sets, 
%    name={\ensuremath{\mathbb{N}}},
%    description={Ensemble des nombres entiers naturels},
%    sort={1}}

\newglossaryentry{e-Nab}{type=sets, 
    name={\ensuremath{\llbracket m, n \rrbracket}},
    description={Ensemble des nombres entiers compris entre $m$ et $n$ inclus},
    sort={1}}

\newglossaryentry{e-R}{type=sets, 
    name={\ensuremath{\mathbb{R}}},
    description={Ensemble des nombres réels},
    sort={2}}

\newglossaryentry{e-Rn}{type=sets, 
    name={\ensuremath{\mathbb{R}^n}},
    description={Ensemble des vecteurs de taille $n$},
    sort={3}}

\newglossaryentry{e-Rmn}{type=sets, 
    name={\ensuremath{\mathbb{R}^{m \times n}}},
    description={Ensemble des matrices de taille $m\times n$},
    sort={4}}



%% Fonctions %%%%%%%%%%%%%%%%%%%%%%%%%%%%%%%%%%%%%%%%%%%%%%%%%%%%%%%%%%

\newglossaryentry{f-signe}{type=functions, 
    name={sgn},
    description={Fonction signe},
    sort={1}}

\newglossaryentry{f-ind}{type=functions, 
    name={\ensuremath{\iota_{\mathcal{A}}}},
    description={Fonction indicatrice sur l'ensemble $\mathcal{A}$},
    sort={2}}


%% Probabilité %%%%%%%%%%%%%%%%%%%%%%%%%%%%%%%%%%%%%%%%%%%%%%%%%%%%%%%%%%

\newglossaryentry{proba-N}{type=probas, 
    name={\parbox{2.8cm}{\ensuremath{x\sim \mathcal{N}(m, \sigma^2)}}},
    description={$x$ suit une loi gaussienne de moyenne $m$ et d'eccart-type $\sigma$},
    sort={1}}

\newglossaryentry{proba-den}{type=probas, 
    name={\ensuremath{P(x)}},
    description={Densité de probabilité de $x$},
    sort={2}}
%\multicolumn{2}{c}{  \sim \mathcal{N}(m, \sigma^2)


%% Dimensions %%%%%%%%%%%%%%%%%%%%%%%%%%%%%%%%%%%%%%%%%%%%%%%%%%%%%%%%%%

\newglossaryentry{P}{
    type=dimensions,
    name={\ensuremath{P}},
    description={Le nombre de pixels},
    sort={01}
}

\newglossaryentry{M}{
    type=dimensions,
    name={\ensuremath{M}},
    description={Le nombre de canaux},
    sort={02}
}

\newglossaryentry{N}{
    type=dimensions,
    name={\ensuremath{N}},
    description={Le nombre de pixels acquis},
    sort={03}
}

\newglossaryentry{Rt}{
    type=dimensions,
    name={\ensuremath{R_{\mathrm{true}}}},
    description={La dimension du véritable sous-espace signal},
    sort={03}
}

\newglossaryentry{R}{
    type=dimensions,
    name={\ensuremath{R}},
    description={La dimension estimée du véritable sous-espace signal},
    sort={04}
}


%% Modèle direct %%%%%%%%%%%%%%%%%%%%%%%%%%%%%%%%%%%%%%%%%%%%%%%%%%%%%%%

\newglossaryentry{r}{
    type=modele,
    name={\ensuremath{r}},
    description={Le rapport d'acquisition $\gls{N}/\gls{P}$},
    sort={01}
}

\newglossaryentry{I}{
    type=modele,
    name={\ensuremath{\mathcal{I}}},
    description={L'ensemble des index des positions spatiales acquises},
    shape={\gls{N}},
    sort={02}
}

\newglossaryentry{Phi}{
    type=modele,
    name={\ensuremath{\Phi}},
    description={Opérateur de sous-échantillonnage tel que $\mathbf{X}_{\mathcal{I}} = \mathbf{X}\Phi$},
    shape={\gls{P}\ensuremath{\times}\gls{N}},
    sort={03}
}

\newglossaryentry{Y}{
    type=modele,
    name={\ensuremath{\mathbf{Y}}},
    description={La matrice qui correspondrait aux données EELS complètes},
    shape={\gls{M} \ensuremath{\times} \gls{P}},
    sort={04}
}

\newglossaryentry{Yi}{
    type=modele,
    name={\ensuremath{\mathbf{Y}_\mathcal{I}}},
    description={La matrice d'observation},
    shape={\gls{M} \ensuremath{\times} \gls{N}},
    sort={05}
}

\newglossaryentry{X}{
    type=modele,
    name={\ensuremath{\mathbf{X}}},
    description={Les données inconnues à reconstruire},
    shape={\gls{M} \ensuremath{\times} \gls{P}},
    sort={06}
}

\newglossaryentry{E}{
    type=modele,
    name={\ensuremath{\mathbf{E}}},
    description={La matrice de bruit},
    shape={\gls{M} \ensuremath{\times} \gls{N}},
    sort={07}
}

\newglossaryentry{sig}{
    type=modele,
    name={\ensuremath{\sigma}},
    description={L'éccart-type du bruit blanc additif gaussien},
    sort={08}
}

\newglossaryentry{Xh}{
    type=modele,
    name={\ensuremath{\hat{\mathbf{X}}}},
    description={L'image reconstruites},
    shape={\gls{M} \ensuremath{\times} \gls{P}},
    sort={09}
}


%% Démélange %%%%%%%%%%%%%%%%%%%%%%%%%%%%%%%%%%%%%%%%%%%%%%%%%%%%%%%%%%%
\newglossaryentry{Mu}{
    type=demelange,
    name={\ensuremath{\mathbf{M}}},
    description={La matrice des composantes spectrales},
    shape={\gls{M} \ensuremath{\times} \gls{P}},
    sort={1}
}

\newglossaryentry{A}{
    type=demelange,
    name={\ensuremath{\mathbf{A}}},
    description={La matrice des abondances},
    shape={\gls{M} \ensuremath{\times} \gls{P}},
    sort={2}
}

%% ACP %%%%%%%%%%%%%%%%%%%%%%%%%%%%%%%%%%%%%%%%%%%%%%%%%%%%%%%%%%%%%%%%%

\newglossaryentry{H}{
    type=acp,
    name={\ensuremath{\mathbf{H}}},
    description={La base des composantes principales associées aux données},
    shape={\gls{M} \ensuremath{\times} \gls{M}},
    sort={1}
}

\newglossaryentry{d}{
    type=acp,
    name={\ensuremath{d^2}},
    description={Les valeurs propres associées aux colonnes de \gls{H}},
    shape={\gls{M}},
    sort={2}
}

\newglossaryentry{S}{
    type=acp,
    name={\ensuremath{\mathbf{S}}},
    description={Les coefficients de représentation des données dans la base \gls{H}},
    shape={\gls{M} \ensuremath{\times} \gls{P}},
    sort={3}
}



% %% 3S %%%%%%%%%%%%%%%%%%%%%%%%%%%%%%%%%%%%%%%%%%%%%%%%%%%%%%%%%%%%%%%%%%

%% Operators
\newglossaryentry{D}{
    type=chap3,
    name={\ensuremath{\mathbf{D}}},
    description={Opérateur de gradient spatial discret avec $P' = m(n-1) + (m-1)n$, où $m$ et $n$ sont respectivement le nombre de lignes et de colonnes de l'image},
    shape={\gls{P} \ensuremath{\times} \gls{P}'},
    sort={01}
}

\newglossaryentry{Delta}{
    type=chap3,
    name={\ensuremath{\Delta}},
    description={Opérateur de laplacien spatial discret, $\Delta = -\mathbf{D}\mathbf{D}^T$},
    shape={\gls{P} \ensuremath{\times} \gls{P}},
    sort={02}
}

% Hyperparamètres
\newglossaryentry{ls2n}{
    type=chap3,
    name={\ensuremath{\lambda_{\mathrm{S2N}}}},
    description={Premier paramètre de la méthode S2N},
    sort={03}
}

\newglossaryentry{ms2n}{
    type=chap3,
    name={\ensuremath{\mu_{\mathrm{S2N}}}},
    description={Premier paramètre de la méthode S2N},
    sort={04}
}

\newglossaryentry{m3s}{
    type=chap3,
    name={\ensuremath{\mu_{\mathrm{3S}}}},
    description={Paramètre de la méthode 3S},
    sort={05}
}

\newglossaryentry{w}{
    type=chap3,
    name={\ensuremath{w}},
    description={Les poids associés à la méthode 3S},
    sort={06}
}

\newglossaryentry{hsig}{
    type=chap3,
    name={\ensuremath{\hat{\sigma}}},
    description={L'éccart-type estimé du bruit blanc additif gaussien},
    sort={07}
}


% %% CLS %%%%%%%%%%%%%%%%%%%%%%%%%%%%%%%%%%%%%%%%%%%%%%%%%%%%%%%%%%%%%%%%%

\newglossaryentry{lcls}{
    type=chap4,
    name={\ensuremath{\lambda_{\mathrm{CLS}}}},
    description={Paramètre de la méthode CLS},
    sort={01}
}


