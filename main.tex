\documentclass[a4paper, twoside, table, justified,
               nofonts, nobib, nohyper, 10pt, french]{tufte-book}

%!TEX root = ../main.tex

%% ==============================================================
%% Encoding

\usepackage[french]{babel}
\usepackage[utf8]{inputenc}
\usepackage[T1]{fontenc}

%% ==============================================================
%% Title page style

%% INPT / Toulouse University title page
\usepackage[ED=MITT-SIAO, Ets=INP]{tlsflyleaf}

% \usepackage{lmodern}\normalfont
% \DeclareFontShape{T1}{lmr}{bx}{sc}{ <-> ssub * cmr/bx/sc }{}
% \DeclareOldFontCommand{\bf}{\normalfont\bfseries}{\mathbf}
% \usepackage[osf,sc]{mathpazo}

%% Setup basic string
\title{Acquisition rapide et reconstruction en imagerie EELS.}
%
\author{Etienne Monier}
%
\defencedate{??/??/????}
%
\lab{Institut de Recherche en Informatique de Toulouse (UMR 5505)}

%% Boss
\nboss{2}
%
\makesomeone{boss}{1}{Nicolas \textsc{Dobigeon}}{Professeur à l'INP-ENSEEIHT}{Directeur de thèse}
\makesomeone{boss}{2}{Thomas \textsc{Oberlin}}{Maître de conférence à l'INP-ENSEEIHT}{Co-directeur de thèse}

%% Referee
\nreferee{2}
\makesomeone{referee}{1}{? \textsc{?}}{?}{Rapporteur}
\makesomeone{referee}{2}{? \textsc{?}}{?}{Rapporteur}

%% Judges
\njudge{7}
\makesomeone{judge}{1}{? \textsc{?}}{?}{Rapporteur}
\makesomeone{judge}{2}{? \textsc{?}}{?}{Rapporteur}
\makesomeone{judge}{3}{? \textsc{?}}{?}{Examinateur}
\makesomeone{judge}{4}{? \textsc{?}}{?}{Examinateur}
\makesomeone{judge}{5}{? \textsc{?}}{?}{Examinateur}
\makesomeone{judge}{6}{Nicolas \textsc{Dobigeon}}{Professeur à l'INP-ENSEEIHT}{Directeur de thèse}
\makesomeone{judge}{7}{Thomas \textsc{Oberlin}}{Maître de conférence à l'ISAE-SUPAERO}{Co-directeur de thèse}


%% ==============================================================
%% Warning filtering

\usepackage{silence}
%\WarningFilter{biblatex}{Patching footnotes failed}
%\WarningFilter{biblatex}{Attempt to redefine deprecated}
\WarningFilter{latex}{Marginpar on page}
\WarningFilter{latexfont}{Font shape}
\WarningFilter{latexfont}{Some font}
\WarningFilter{latexfont}{Size substitutions}

%% ==============================================================
%% Index

% Generates the index https://en.wikibooks.org/wiki/LaTeX/Indexing
% https://www.overleaf.com/learn/latex/Indices
% To make clickable links with hyperref
% pass nohyper option to tufte documentclass
% and load hyperref after imakeidx
\usepackage{imakeidx}
\indexsetup{level=\chapter*, toclevel=chapter}
\makeindex[title=Index, columns=1, intoc]

%% ==============================================================
%% Hyperef links

\usepackage[svgnames]{xcolor}
\definecolor{mydarkblue}{rgb}{0,0.08,0.45}
\definecolor{myblue}{RGB}{18,75,126}
\definecolor{burgundy}{RGB}{128,0,32}

\usepackage{hyperref}
\hypersetup{
    linktoc=all,
    breaklinks=true,
    colorlinks=true,
    linkcolor=mydarkblue,
    citecolor=mydarkblue,
    filecolor=mydarkblue,
    urlcolor=mydarkblue
}


%% ==============================================================
%% Glossary

% https://www.overleaf.com/learn/latex/Glossaries
% Terminal: makeglossaries main
% rebuild main.tex
% To make clickable links with hyperref
% load hyperref before glossaries

% Some lengths
\TufteRecalculate
\newlength\fullwidthwidth
\makeatletter\setlength\fullwidthwidth{\@tufte@fullwidth}\makeatother

\newlength\centralcol
\setlength\centralcol{\fullwidthwidth}
\addtolength{\centralcol}{-5cm}

% Glossaries input.
\usepackage[%
    % nopostdot,  % If no final point is desired for description.
    nonumberlist,  % The location should not be displayed.
    acronym,
    toc,
    section=subsection,  % Sets the printglossaries be a section.
    numberedsection=false,  % To have section* instead of section.
    nogroupskip=true,
    xindy,
    ucmark,
    shortcuts]{glossaries}

%
% New gloassaries
%

% Adds a key to glossaries entries
\glsaddstoragekey{shape}{}{\glsshape}

% Add all entries into list


% Glossaries sections
\newglossary[nglg]{notgen}{nge}{ngtn}{Notations générales}
\newglossary[nlg]{notation}{not}{ntn}{Notations}

%
% Format
%

% Defines the accronym display style. Long desc. first, then short one.
\setacronymstyle{long-short}

% Removes space after section name
\renewcommand{\glossarypreamble}{\vspace*{-\baselineskip}}

% This command disables hyperlinks from text to glossary list.
\glsdisablehyper

% https://tex.stackexchange.com/questions/269565/glossaries-how-to-customize-list-of-symbols-with-additional-column-for-units

%\setlength\centralcol{10cm minus 1.5cm minus 1.5cm}
\newglossarystyle{symbunitlong}{%
    \setglossarystyle{long3col}% base this style on the list style
    \renewenvironment{theglossary}{% Change the table type --> 3 columns
        \begin{longtable}{p{1.5cm}p{\centralcol}>{\hfill}p{1.5cm}}}%
        {\end{longtable}}%
    %
%    \renewcommand*{\glossaryheader}{%  Change the table header
%        \bfseries Sign & \bfseries Description & \bfseries Unit \\
%        \hline
%        \endhead}
    \renewcommand*{\glossentry}[2]{%  Change the displayed items
        \glstarget{##1}{\glossentryname{##1}} %
        & \glossentrydesc{##1}% Description
        & \glsshape{##1}  \tabularnewline
    }
}

\newglossarystyle{acrolong}{%
    \setglossarystyle{long3col}% base this style on the list style
    \renewenvironment{theglossary}{% Change the table type --> 3 columns
        \begin{longtable}{p{1.5cm}p{\textwidth-1.5cm}}}%
        {\end{longtable}}%
    %
    %    \renewcommand*{\glossaryheader}{%  Change the table header
    %        \bfseries Sign & \bfseries Description & \bfseries Unit \\
    %        \hline
    %        \endhead}
    \renewcommand*{\glossentry}[2]{%  Change the displayed items
        \glstarget{##1}{\glossentryname{##1}} %
        & \glossentrydesc{##1}% Description
        \tabularnewline
    }
}


%
% Starts glossaries and input entries
%

\makeglossaries

%!TEX root = ../main.tex
% To build the glossary: makeglossaries main

%%%%% Acronyms
% \newacronym[plural={<plural acronym>},
%             first={<text displayed at first occurrence>},
%             firstplural={<idem, with plural>}]
%             {<label>}
%             {<acronym>}
%             {<full name to display in acronym section>}

% Laboratoires
\newacronym{irit}
           {IRIT}
           {Institut de Recherche en Informatique de Toulouse}

\newacronym{lps}
           {LPS}
           {Laboratoire de Physique des Solides}

% Microscopes
\newacronym{stem}
           {STEM}
           {Scanning Transmission Electron Microscope}

\newacronym{tem}
           {TEM}
           {Transmission Electron Microscope}

\newacronym{sem}
           {SEM}
           {Scanning Electron Microscope}

\newacronym{mfa}
           {MFA}
           {microscopie à force atomique}

% Modalités
\newacronym{haadf}
           {HAADF}
           {High-Angle Annular Dark-Field}

\newacronym{eels}
           {EELS}
           {Electron Energy Loss Spectroscopy}

\newacronym{edx}
           {EDX}
           {analyse dispersive en énergie}


% Techniques
\newacronym{pca}
           {ACP}
           {Analyse par Composantes Principales}

\newacronym{cs}
           {CS}
           {Compressed Sensing}

\newacronym{ppv}
           {PPV}
           {Plus Proches Voisins}

\newacronym{ad}
           {AD}
           {Apprentissage de Dictionnaire}

\newacronym{ebi}
           {EBI}
           {Exemplar-Based Inpainting}

\newacronym{mc}
           {MC}
           {moindre carré}
        
\newacronym{dct}
           {DCT}
           {transformée en cosinus discrète}

\newacronym{tv}
           {TV}
           {variation totale}

%%%% Glossary entries


%% Notations génériques

\newglossaryentry{g-a}{type=notgen, name={\ensuremath{a}}, description={Scalaire}, sort={01}}

\newglossaryentry{g-av}{type=notgen,
	name={\ensuremath{\mathbf{a}}},
	description={Vecteur colonne}, 
	sort={02}}

\newglossaryentry{gavi}{type=notgen, 
	name={\ensuremath{\mathbf{a}_i}}, 
	description={$i^\text{ème}$ composante du vecteur \gls{g-av}}, 
	sort={03}}

\newglossaryentry{g-A}{type=notgen, 
	name={\ensuremath{\mathbf{A}}},
	description={Matrice},
	sort={04}}

\newglossaryentry{g-Aij}{type=notgen,
	name={\ensuremath{\mathbf{A}_{ij}}},
	description={Coefficient $(i, j)$ de la matrice \gls{g-A}},
	sort={05}}

\newglossaryentry{g-Aj}{type=notgen, 
	name={\ensuremath{\mathbf{A}_j}},
	description={$j^\text{ème}$ colonne de la matrice \gls{g-A}},
	sort={06}}

\newglossaryentry{g-Ai}{type=notgen, 
	name={\ensuremath{\mathbf{A}_{i, :}}},
	description={$i^\text{ème}$ ligne de la matrice \gls{g-A}},
	sort={07}}

\newglossaryentry{g-T}{type=notgen, 
	name={\ensuremath{(\cdot)^{T}}},
	description={Transposée},
	sort={08}}

\newglossaryentry{g-pm}{type=notgen, 
	name={\ensuremath{\mathbf{AB}}},
	description={Produit matriciel},
	sort={09}}

\newglossaryentry{g-nf}{type=notgen, 
	name={\ensuremath{||\mathbf{A}||_F}},
	description={Norme de Frobenius de \gls{g-A}},
	sort={10}}



%% Dimensions
\newglossaryentry{P}{
	type=notation,
	name={\ensuremath{P}},
	description={Le nombre de pixels},
	sort={01}
}

\newglossaryentry{M}{
	type=notation,
	name={\ensuremath{M}},
	description={Le nombre de cannaux},
	sort={02}
}

\newglossaryentry{N}{
	type=notation,
	name={\ensuremath{N}},
	description={Le nombre de pixels acquis},
	sort={03}
}

\newglossaryentry{r}{
	type=notation,
	name={\ensuremath{r}},
	description={Le rapport d'acquisition $\gls{N}/\gls{P}$},
	sort={04}
}

%% Variance
\newglossaryentry{sig}{
	type=notation,
	name={\ensuremath{\sigma}},
	description={La variance du bruit blanc additif gaussien},
	sort={05}
}

%% Sets
\newglossaryentry{I}{
	type=notation,
	name={\ensuremath{\mathcal{I}}},
	description={L'ensemble des index des positions spatiales acquises},
	shape={(\gls{N}, )},
	sort={06}
}

%% Matrices
\newglossaryentry{X}{
	type=notation,
	name={\ensuremath{\mathbf{X}}},
	description={Les données inconnues à restituer},
	shape={(\gls{M}, \gls{P})},
	sort={07}
}

\newglossaryentry{Y}{
	type=notation,
	name={\ensuremath{\mathbf{Y}}},
	description={La matrice d'observation},
	shape={(\gls{M}, \gls{N})},
	sort={08}
}

\newglossaryentry{B}{
	type=notation,
	name={\ensuremath{\mathbf{B}}},
	description={La matrice de bruit gaussien},
	shape={(\gls{M}, \gls{N})},
	sort={09}
}

\newglossaryentry{Xh}{
    type=notation,
    name={\ensuremath{\hat{\mathbf{X}}}},
    description={Les données reconstruites},
    shape={(\gls{M}, \gls{P})},
    sort={10}
}





%% ==============================================================
%% Nomenclature

% to build nomenclature
% makeindex main.nlo -s nomencl.ist -o main.nls
% rebuild main.tex
\usepackage[intoc]{nomencl}
\makenomenclature
% \nomlabelwidth=30mm
\setlength{\nomitemsep}{.5\parsep}
\renewcommand{\nomname}{Notations}
%\newcommand{\nomunit}[1]{%
%\renewcommand{\nomentryend}{\hspace*{\fill}#1}}

\usepackage{etoolbox}  % required in nomenclature.tex

%% ==============================================================
%% Subappendix

% Add appendix to each chapter
\usepackage[toc,page]{appendix}
\usepackage{chngcntr}

\AtBeginEnvironment{subappendices}{%
    \newpage
    \section*{Appendices}
    \phantomsection
    \addtocontents{toc}{\vspace{1ex}} % small vertical space
    \addtocontents{toc}{\protect\contentsline{section}{\protect\textsc{Appendices}}{}{}}
    \counterwithin{equation}{section}
    \counterwithin{figure}{section}
    \counterwithin{table}{section}
}
\AtEndEnvironment{subappendices}{%
    \counterwithin{equation}{section}
    \counterwithout{figure}{section}
    \counterwithout{table}{section}
}

%% ==============================================================
%% Style de page

\renewcommand\chaptermark[1]{\markboth{\chaptername\thechapter. #1}{}}
\renewcommand\sectionmark[1]{\markright{\thesection. #1}}

\usepackage[fit]{truncate}
\newcommand{\markformat}[1]{\truncate{0.95\textwidth}{\footnotesize\scshape\nouppercase{#1}}}

\renewcommand\frontmatter{%
    \cleardoublepage%
    \pagenumbering{roman}%
    %\pagestyle{plain}%
    \fancyhf{}%
    \ifthenelse{\boolean{@tufte@twoside}}%
    {\fancyhead[LE,RO]{\thepage}}%
    {\fancyhead[RE,RO]{\thepage}}%
}

\renewcommand\mainmatter{%
    \cleardoublepage%
    \pagenumbering{arabic}%
    \fancyhf{}%
%    \renewcommand{\chaptermark}[1]{\markboth{##1}{}}%
%    \fancyhead[LE]{\thepage\quad\smallcaps{\newlinetospace{\plaintitle}}}% book title
%    \fancyhead[RO]{\smallcaps{\newlinetospace{\leftmark}}\quad\thepage}% chapter title
\fancyhead[LO]{\markformat{\rightmark}}
\fancyhead[RE]{\markformat{\leftmark}}
\fancyhead[LE, RO]{\footnotesize\thepage}
}

\usepackage{fancyhdr}
% \pagestyle{fancy}

\RequirePackage{etoolbox}                                                   
\appto\frontmatter{\pagestyle{fancy}}                                       
\appto\mainmatter{\pagestyle{fancy}}                                        
\appto\backmatter{\pagestyle{empty}} 

% \fancyhf{}



%% ==============================================================
%% Text packages

\usepackage{xcolor}  % Colors
\usepackage{enumerate}
\usepackage[shortlabels]{enumitem}       % personnalisation des enumerate
\setlist[itemize]{label=$\square$}  % black

%% ==============================================================
%% Graphics packages

% Figure / Float
% \usepackage{float}      % Ability to define new figure style, proposes H as a position.
% \usepackage[caption=false]{subfig}     % Create subfloats with \subfloat[caption]{figure}
% or \subtable[caption]{figure}
% \usepackage{subfloat}
% \usepackage[export]{adjustbox}  % To use \vphantom for vertical alignment of subfloat.


% Include graphic
\usepackage{graphicx}
\DeclareGraphicsExtensions{.pdf,.jpg,.png}
\graphicspath{{img/}}

\usepackage[center]{subfigure}

% Tikz config
\usepackage{style/tikzstyle}

%% ==============================================================
%% Math packages

%% Math symbols and font
\usepackage{amsmath}    % Main commands.
\usepackage{amssymb}    % Main symbols
\usepackage{dsfont}     % \mathds{1} pour indicatrice
\usepackage{amsfonts}
\usepackage{amsthm}
\usepackage{mathrsfs}   % Ralph Smith’s Formal Script Font : mathscr{A}
\usepackage{mathtools}  % \DeclarePairedDelimiter{\ceil}{\lceil}{\rceil}
\usepackage{stmaryrd}   % \llbracket et \rrbracket %sinon : $[\![$ et $]\!]$

%% Algorithms
\usepackage[ruled,vlined]{algorithm2e} % package environnement algorithme

%% Theorems and definitions
\newtheorem{mydef}{Définition}

%\usepackage{tcolorbox}
%\tcbuselibrary{skins}
%\tcbuselibrary{theorems}
%
%\newtcbtheorem[number within=chapter]{definition}{Définition}{%
%colbacktitle=gray!40, colback=gray!20, boxrule=0pt, fonttitle=\bfseries,%
%arc=0pt,outer arc=0pt%
%}{definitionlabel}


%% ==============================================================
%% Tabular packages

\usepackage{tabularx}
\usepackage{multirow}
\usepackage{booktabs}     % For serious tables (\toprule, \midrule, \bottomrule).


%% ==============================================================
%% FontAwesome

\usepackage{pgffor}
\usepackage{fontawesome}

\newcommand{\minusfa}[1][1]{%
    \foreach \n in {1,...,#1}{\color{red}\faicon{minus-square}}%
}
\newcommand{\plusfa}[1][1]{%
    \foreach \n in {1,...,#1}{\color{green}\faicon{plus-square}}%  plus plus-square plus-circle
}
\newcommand{\checkfa}{{\color{black}\faicon{check-square}}}  % check check-square check-circle

%% ==============================================================
%% Pretty refs

\usepackage[noabbrev]{cleveref}


\usepackage[]{todo}
\renewcommand\todoformat{\bfseries\small\color{red}}

\newcommand{\mref}{{\bfseries\color{red}[REF]}}
% \usepackage{showframe}


%% ==============================================================
%% Correct label

%\let\oldlabel\label
%\renewcommand{\label}[1]{\protect\oldlabel{#1}}
%
%% margincaption, leftbody, ragged, wide
%\usepackage[rightcaption, ragged, margincaption]{sidecap}
%\sidecaptionvpos{figure}{t} 
%\sidecaptionvpos{table}{t}
%
%\usepackage{floatrow}
%\DeclareFloatSeparators{mcapwidth}{\begingroup\hspace{\marginparsep}\endgroup}
%\floatsetup[widefigure]{margins=hangright,capposition=beside,
%    capbesideposition={top,right},floatwidth=\textwidth, capbesidewidth=\marginparwidth}, capbesidesep=mcapwidth}
%


\usepackage{lipsum}
\usepackage{comment}

\usepackage{showframe}


\begin{document}

%%%%%%%%%%%%%%%%%%%%%%%%%%%%%%%%%%%%%%%%%%%%%%%%%%%%%%%%%%%%%%%%%%%%%%%%%%%%
% The front matter contains title page, acknowledgements, toc, nomenclature and dedication
%%%%%%%%%%%%%%%%%%%%%%%%%%%%%%%%%%%%%%%%%%%%%%%%%%%%%%%%%%%%%%%%%%%%%%%%%%%%

\makeflyleaf
\frontmatter

\setcounter{chapter}{-1}
\setcounter{secnumdepth}{3}
\setcounter{tocdepth}{1}


%
% Remerciements
%

%!TEX root = ../main.tex

\chapter*{Remerciements} % (fold)
\label{ch:remerciements}

    Je remercie chalereusement \dots
    
    \lipsum[1-3]

% chapter remerciements (end)


%
% Dédicace
%

%!TEX root = ../main.tex

\clearpage
~\vfill
\thispagestyle{plain}
\begin{doublespace}
    \noindent\fontsize{18}{22}\selectfont\itshape
    \nohyphenation
    To ...
\end{doublespace}
\vfill
\vfill


%
% Table des matières
%

\begin{fullwidth}
    \tableofcontents
\end{fullwidth}


%
% Notations
%

%!TEX root = ../main.tex
% to build nomenclature
% makeindex main.nlo -s nomencl.ist -o main.nls
% rebuild main.tex

% A-Z sets the category
% 0x sets the order items will be displayed

\renewcommand\nomgroup[1]{%
  \item[\bfseries
      \ifstrequal{#1}{A}{Notations génériques}{%
      \ifstrequal{#1}{B}{Number Sets}{%
      \ifstrequal{#1}{C}{Other Symbols}{}}}%
    ]
}

%
% Notations génériques
%
\nomenclature[A, 01]{$a$}{Scalaire}

\nomenclature[A, 02]{$\mathbf{a}$}{Vecteur colonne}
\nomenclature[A, 03]{$\mathbf{a}_i$}{$i^\text{ème}$ composante du vecteur $\mathbf{a}$}

\nomenclature[A, 04]{$\mathbf{A}$}{Matrice}
\nomenclature[A, 05]{$\mathbf{A}_{ij}$}{Coefficient $(i, j)$ de la matrice $\mathbf{A}$}
\nomenclature[A, 06]{$\mathbf{A}_j$}{$j^\text{ème}$ colonne de la matrice $\mathbf{A}$}
\nomenclature[A, 07]{$\mathbf{A}_{i, :}$}{$i^\text{ème}$ ligne de la matrice $\mathbf{A}$}

\nomenclature[A, 08]{$(\cdot)^{T}$}{Transposée}
\nomenclature[A, 09]{$\mathbf{A}\mathbf{B}$}{Produit matriciel}

\nomenclature[A, 10]{$||\mathbf{A}||_F$}{Norme de Frobenius de $\mathbf{A}$} 


%\nomenclature[A, 02]{$c$}{Speed of light in a vacuum inertial system
%  \nomunit{$299,792,458\, m/s$}}
%\nomenclature[A, 03]{$h$}{Plank Constant
%  \nomunit{$6.62607 \times 10^{-34}\, Js$}}

%\nomenclature[B, 01]{$\mathbb{H}$}{Quaternions}
%\nomenclature[B, 02]{$\mathbb{C}$}{Complex Numbers}
%\nomenclature[B, 03]{$\mathbb{R}$}{Real Numbers}
%
%\nomenclature[C]{$V$}{Constant Volume}
%\nomenclature[C]{$\rho$}{Friction Index}
%
%\begin{fullwidth}
%  \printnomenclature
%\end{fullwidth}


%
% Liste des acronymes
%
\begin{fullwidth}
    % \printnomenclature
    \chapter*{Notations}
    \addcontentsline{toc}{chapter}{Notations}
    \glsaddall
    \printglossary[type=notgen,style=symbunitlong]
    \printglossary[type=notation,style=symbunitlong]
    \printglossary[type=\acronymtype, style=acrolong]
\end{fullwidth}


%%%%%%%%%%%%%%%%%%%%%%%%%%%%%%%%%%%%%%%%%%%%%%%%%%%%%%%%%%%%%%%%%%%
% The main matter contains numbered chapter: introduction, chapters
%%%%%%%%%%%%%%%%%%%%%%%%%%%%%%%%%%%%%%%%%%%%%%%%%%%%%%%%%%%%%%%%%%%
\mainmatter
\setcounter{tocdepth}{2}

%
% Introduction générale
%

%!TEX root = ../main.tex

\chapter{Introduction}
\label{ch:introduction}

Ce document constitue la synthèse de mon travail de thèse 
    
    Présenter succinctement le contexte de la thèse, la collaboration avec le LPS, le travail effectué, la motivation, les résultats.


% chapter introduction (end)
  % Chapter 0

%
% Introduction et contexte
%

\begin{fullwidth}
	\part{Introduction et contexte de l'étude}
\end{fullwidth}

%!TEX root = ../main.tex

\chapter{L'imagerie \glsentryshort{stem} : présentation et problématiques}
\label{ch-chapter_1}

    \dochaptoc
    \graphicspath{{img/chapitre2/}}

    \section{Présentation de la microscopie \glsentryshort{stem}}

    L'ensemble du travail de thèse présenté dans ce manuscrit se base sur des données issues d'un système de microscopie appelé Microscope \'Electronique en Transmission à Balayage ou encore \acf{stem}\glsunset{stem}. Ce système utilise un faisceau d'électrons pour illuminer un échantillon et obtenir ainsi un agrandissement de la zone à étudier, il s'agit donc d'un microscope électronique. Nous allons commencer par positionner le \gls{stem} au sein de la microscopie électronique pour ensuite décrire plus précisément son fonctionnement.

    \subsection{La microscopie électronique}

    Un microscope électronique se compose classiquement d'une source d'électrons (aussi appelé canon), de lentilles électromagnétiques permettant de focaliser le faisceau de particules sur l'échantillon et d'un détecteur. En définitive, les types de microscopes électroniques se différencient par la façon dont l'échantillon est illuminé et par la position du détecteur. Nous y trouvons donc :
    \begin{itemize}
    	\item la microscopie électronique en transmission (aussi appelée \gls{tem}) pour laquelle le faisceau d'électrons illumine l'ensemble de l'échantillon et le traverse pour être ensuite analysé,
    	\item la microscopie \gls{stem} pour laquelle le faisceau d'électrons est focalisé sur un point de l'échantillon et le traverse pour être ensuite analysé,
    	\item la microscopie électronique à balayage (aussi appelée \gls{sem}) pour laquelle le faisceau d'électrons est focalisé sur un point de l'échantillon et le détecteur positionné en amont capte des électrons secondaires s'étant déviés.
    \end{itemize}
    La \cref{fig-chap2-micros-electron} propose un schéma simplifié de ces configurations. Chacune de ces modalités permettent l'acquisition d'une image contrastée 2D qui est une représentation agrandie de l'échantillon.

    \begin{figure*}[t!]
       	\centering
       	\subfigure[Légende]{\includegraphics[]{img/chapitre2/figure1/subfig-a/electronic-legend.pdf}
       	}\\
       	\subfigure[Le microscope \gls{tem}]{\includegraphics[]{img/chapitre2/figure1/subfig-b/electronic-tem.pdf}}
       	\hspace*{2cm}
       	%
       	\subfigure[Le microscope \gls{stem}]{\includegraphics[]{img/chapitre2/figure1/subfig-c/electronic-stem.pdf}}
       	\hspace*{2cm}
       	%
       	\subfigure[Le microscope \gls{sem}]{\includegraphics[]{img/chapitre2/figure1/subfig-d/electronic-sem.pdf}}
       	%
           \vspace{1em}
       	\caption{Schéma de principe des différents types de microscopie électronique.%
               \protect\label{fig-chap2-micros-electron}}
   \end{figure*}

    Les microscopes \gls{stem} et \gls{sem} se différencient du \gls{tem} puisque le faisceau balaye l'échantillon ligne par ligne au lieu de l'illuminer entièrement. Il en résulte que ces premiers sont destinés principalement à la \emph{spectroscopie}, i.e. à l'acquisition d'un spectre pour chaque position spatiale. Les données peuvent alors être perçues comme un cube ayant deux directions spatiales et une direction spectrale. Des applications classiques de ces systèmes sont la détection et cartographie d'éléments chimiques présents dans l'échantillon (cf. \cref{sec-exploitation-eels}). Le \gls{tem} est davantage utilisé en \emph{imagerie} pour représenter l'échantillon sans analyse chimique possible.

    La suite de ce manuscrit se focalisera sur la microscopie \gls{stem} qui est le centre de notre étude. C'est pourquoi le principe physique de la microscopie en transmission va être évoquée et les modalités d'acquisition classiques vont être présentées. A cette fin, un schéma plus détaillé de ce système et de ses modalités d'acquisition est donné en \cref{fig-chap2-stem-detail} et servira de support. D'autre part, le contenu technique de ce chapitre s'inspire du livre de Egerton~\cite{egerton2011electron} et nous renvoyons les lecteurs curieux à cet ouvrage pour de plus amples informations.

    \begin{figure*}[htbp]
    	\centering
    	\includegraphics[]{img/chapitre2/figure2/stem-detail.pdf}
    	\caption{Un schéma de principe détaillé du \gls{stem}.
        	\protect\label{fig-chap2-stem-detail}}
    \end{figure*}

    \subsection{Principe de la microscopie en transmission}

    La microscopie en transmission étudie les interactions physiques entre un faisceau d'électron et la matière qu'il traverse. Pour ce faire, un canon fournit une certaine énergie cinétique à un faisceau d'électrons, celui-ci est alors focalisé en un point de l'échantillon à l'aide de lentilles électromagnétiques. Afin que ce faisceau \emph{traverse} l'échantillon, plusieurs conditions doivent être réunies :
    \begin{enumerate}[label=(\alph*)]
    	\item l'énergie cinétique des électrons doit être suffisante (typiquement 100keV),
    	\item l'échantillon doit être suffisamment fin (typiquement 100nm pour un faisceau de 100keV).
    \end{enumerate}
    Dans ces conditions, les électrons traversent l'échantillon sans subir d'absorption ni de réflexion notoire. A noter qu'il est nécessaire de faire le vide dans le corps du microscope afin d'éviter toute collision entre le faisceau et les molécules constituant notre atmosphère. Dès lors, des interactions électron-atome vont avoir lieu \emph{au sein de l'échantillon}, séparés grossièrement en trois catégories : diffusion élastique, diffusion inélastique en couche électronique basse et diffusion inélastique en couche électronique haute.

    La diffusion élastique se caractérise par une interaction sans perte d'énergie cinétique. Elle fait intervenir une répulsion électrostatique entre le noyau électronique fortement chargé et l'électron incident. Le champ électrostatique est puissant au c\oe{}ur de l'atome et l'électron incident est d'autant plus dévié de sa trajectoire qu'il passe à proximité du noyau (cf \cref{fig-chap2-interactions-a}). Cela dit, la majorité des électrons traversent l'atome suffisamment loin du noyau pour être peu déviés et l'angle de diffusion n'est généralement que de quelques degrés (10-100 mrad).

    Une diffusion inélastique, quant à elle, se caractérise par une perte d'énergie cinétique. L'énergie perdue permet alors de déplacer un ou plusieurs électrons appartenant à l'atome sur des couches électroniques plus hautes. Dans le cas où l'électron atomique se situe en couche électronique basse, l'énergie transférée correspond à la différence de niveau d'énergie entre les niveaux de départ et d'arrivée (cf \cref{fig-chap2-interactions-b}). Dans le cas où l'électron transféré se situe en couche électronique haute, la quantité d'énergie transférée peut prendre un continuum de valeur puisque l'électron se déplace au sein d'une bande de valence (cf \cref{fig-chap2-interactions-c}).

    Il en résulte que les électrons traversant l'échantillon sont à la fois déviés de leur trajectoire (possiblement fortement) et ralentis dû à un transfert d'énergie. Les modalités d'acquisition du \gls{stem} se basent sur la détection de ces deux effets.

    \begin{figure}
    	\centering
    	\subfigure[Diffusion élastique]{
    \begin{tikzpicture}[scale=0.5]
        \clip (-2.5, -4) rectangle (2.5, 4);
        \coordinate (O) at (0, 0);
        \coordinate (A) at (0.4, 4);
        \coordinate (B) at (1.9, 4);

        % Circles
        \draw (O) circle (1cm);
        \draw (O) circle (2cm);

        % Central atom
        \fill[black] (O) circle (0.1cm);

        % Atoms on layers # 1
        \fill[black] (0:1) circle (0.1cm);
        \fill[black] (180:1) circle (0.1cm);

        % Atoms on layers # 2
        \fill[black] (45:2) circle (0.1cm);
        \fill[black] (90+45:2) circle (0.1cm);
        \fill[black] (180+45:2) circle (0.1cm);
        \fill[black] (270+45:2) circle (0.1cm);

        % Incident atoms
        \fill[black] (A) circle (0.1cm);
        \fill[black] (B) circle (0.1cm);

        \draw[->, thick ] (A) -- (0.4, 0) arc (0:-140:0.4) -- ++ (140:3);
        \draw[->, thick, rounded corners=1cm ] (B) -- ++(0, -4.5) -- (-2.5, -3);

    \end{tikzpicture}
    \label{fig-chap2-interactions-a}
}\quad
\subfigure[Diffusion inélastique en couche électronique basse]{
    \begin{tikzpicture}[scale=0.5]
        \clip (-2.5, -4) rectangle (2.5, 4);

        \coordinate (O) at (0, 0);
        \coordinate (A) at (1, 4);

        % Circles
        \draw (O) circle (1cm);
        \draw (O) circle (2cm);

        % Central atom
        \fill[black] (O) circle (0.1cm);

        % Atoms on layers # 1
        \filldraw[draw=black, fill=white] (0:1) circle (0.1cm);
        \fill[black] (180:1) circle (0.1cm);
        \draw (120:1) node [above left] {K};

        % Atoms on layers # 2
        \fill[black] (45:2) circle (0.1cm);
        \fill[black] (90+45:2) circle (0.1cm);
        \fill[black] (180+45:2) circle (0.1cm);
        \fill[black] (270+45:2) circle (0.1cm);
        \draw (120:2) node [above left] {L};

        % Incident atoms
        \fill[black] (A) circle (0.1cm);
        \draw[->, thick, rounded corners=0.7cm ] (A) -- (1, 0) -- (-2.5, -3.5);

        % Other atom and arrows
        \fill[black] (2, 0) circle (0.1cm);
        \draw [->] (1.2, 0.1) -- ++ (0.6, 0);
        \draw [<-, dashed] (1.2,- 0.1) -- ++ (0.6, 0);

    \end{tikzpicture}
    \label{fig-chap2-interactions-b}
}\quad
\subfigure[Diffusion inélastique en couche électronique haute]{
    \begin{tikzpicture}[scale=0.5]
        \clip (-3.25, -4) rectangle (3.25, 4);

        \coordinate (O) at (0, 0);
        \coordinate (A) at (1.4, 4);

        % Circles
        \draw (O) circle (1cm);
        \draw (O) circle (2cm);

        % Central atom
        \fill[black] (O) circle (0.1cm);

        % Atoms on layers # 1
        \fill[black] (0:1) circle (0.1cm);
        \fill[black] (180:1) circle (0.1cm);

        % Atoms on layers # 2
        \filldraw [draw=black, fill=white] (45:2) circle (0.1cm);
        \fill[black] (90+45:2) circle (0.1cm);
        \fill[black] (180+45:2) circle (0.1cm);
        \fill[black] (270+45:2) circle (0.1cm);

        % Layer # 3
        \fill [black!15, even odd rule] (O) circle (3.25cm) circle (2.75cm);
        \draw [densely dashed] (O) circle (3cm);

        % Incident atoms
        \fill[black] (A) circle (0.1cm);
        \draw[->, thick, rounded corners=1cm ] (A) -- ++(0, -2.55) -- (-1.5, -3.5);

        % Other atom and arrows
        \fill [black] (45:3) circle (0.1cm);
        \draw [->, rotate=45] (2.2, 0.1) -- ++ (0.6, 0);
        \draw [<-, dashed, rotate=45] (2.2,- 0.1) -- ++ (0.6, 0);

    \end{tikzpicture}
    \label{fig-chap2-interactions-c}
}

        \vspace{1em}
    	\caption{Une représentation classique de la diffusion électronique inspirée de~\cite{egerton2011electron}.%
            \protect\label{fig-chap2-interactions}}
    \end{figure}


    \subsection{Les modalités d'acquisition en microscopie \glsentryshort{stem}}

    Le \gls{stem} permet entre autre trois modalités d'acquisition : la cathodoluminescence, l'acquisition fond noir angulaire à grand angle (ou encore \gls{haadf}) et la spectroscopie par perte d'énergie (ou encore \gls{eels}). La \cref{fig-chap2-stem-detail} sert de support pour situer ces instruments dans le \gls{stem}.

    \paragraph*{Cathodoluminescence} Pour certains échantillons, le faisceau incident produit un phénomène de fluorescence dépendant de la nature de l'échantillon et de ses défauts. La technique consistant à acquérir ce signal et à l'étudier s'appelle \emph{la cathodoluminescence}. Le flux lumineux est recueilli à l'aide d'un miroir situé en amont de l'échantillon puis envoyé vers un spectromètre qui procède à l'analyse.

    \paragraph{\gls{haadf}} Le faisceau incident se situe dans l'axe optique de l'appareil et une grande partie des électrons traversent l'échantillon en demeurant dans cet axe, la diffusion élastique est alors négligeable. Néanmoins, une partie des électrons sont significativement déviés de l'axe optique en sortie.  Un capteur annulaire placé en aval de l'échantillon capte ces électrons et délivre un signal proportionnel. Cette technique appelée \gls{haadf} permet de fournir une image 2D de l'échantillon. Un exemple d'acquisition \gls{haadf} est fournie à la \cref{fig-chap2-haadf-ex}.

    \begin{figure}%[htbp]
    	\centering
    	\subfigure[\label{fig-chap2-haadf-ex-a}Acquisition basse-résolution]%
            {\includegraphics[height=0.25\textwidth]{img/chapitre2/figure4/haadf-LR.png}}
        \hspace{1em}
        \subfigure[\label{fig-chap2-haadf-ex-b}Acquisition haute-résolution]%
            {\includegraphics[height=0.25\textwidth]{img/chapitre2/figure4/haadf-HR-sc.png}}
     	%
        \caption{\protect\label{fig-chap2-haadf-ex}Exemples d'acquisitions \gls{haadf}. L'acquisition basse-résolution~\protect\subref{fig-chap2-haadf-ex-a} est à échelle micrométrique  tandis que l'acquisition haute-résolution~\protect\subref{fig-chap2-haadf-ex-b} est à échelle nanométrique.}
     \end{figure}


    \paragraph*{\gls{eels}} Comme expliqué précédemment, une quantité importante des électrons traversant l'échantillon  perdent une partie de leur énergie initiale. Afin d'étudier cela, le faisceau demeuré dans l'axe optique frappe un spectromètre séparant les électrons en fonction de leur énergie. Une caméra CCD permet de compter, pour chaque position sur l'échantillon, la quantité d'électrons ayant conservé une énergie donnée. \`A chaque position spatiale correspond alors un spectre de perte d'énergie, les images \gls{eels} sont d'ailleurs également appelées \emph{spectre-image}. La microscopie \gls{eels} constitue le centre de cette étude, c'est pourquoi nous allons détailler ces données et leurs propriétés dans la section suivante.


    \section{Propriétés des données \glsentryshort{eels}}\label{sec-prop-eels}

    \begin{figure}[t]
        \centering
        \includegraphics[]{img/chapitre2/figure5/eels-spectrum-shape.pdf}
        \caption{Représentation d'un spectre \gls{eels}.}
        \label{fig-chap2-eels-spectrum-shape}
    \end{figure}

    \paragraph{Caractéristiques spectrales} La forme générique d'un spectre \gls{eels} est représenté à la \cref{fig-chap2-eels-spectrum-shape}. Celui-ci représente le nombre d'électrons ayant traversé l'échantillon en fonction de l'énergie perdue après la traversée (on parle de \emph{canal} pour désigner l'indice associé à une perte d'énergie). Il se compose de trois parties :
    \begin{itemize}
    	\item un pic de perte nulle qui correspond à l'ensemble des électrons ayant traversé l'échantillon sans interagir avec lui, et donc sans avoir perdu d'énergie,
    	\item un pic plus étalé correspondant à une excitation collective de la bande de valence,
    	\item une zone d'intérêt où un ensemble de pics caractéristiques (appelés \emph{seuils}) émergent du fond décroissant.
    \end{itemize}
    \`A noter que le pic de pertes nulles permet de calibrer le zéro de l'axe des abscisses après acquisition. \emph{L'information utile est porté par la position, la forme et l'amplitude des seuils}. En effet, la position d'un seuil permet de déterminer l'élément présent dans l'échantillon tandis que son amplitude nous renseigne sur son abondance pour chaque position spatiale. Enfin, la forme du seuil peut varier suivant la configuration électronique de l'élément étudié.
    %
    Pour servir d'exemple, les seuils généralement rencontrés dans nos données se situent à des pertes d'énergie de l'ordre de 500 à 1000 eV (pour rappel, une énergie typique en amont de l'échantillon est 100keV).
    %
    Un exemple réel de spectre est donné en \cref{fig-chap2-eels-real-spectra-figure}.


    \begin{figure}[htbp]
        \centering
        \subfigure[La position des spectres]{
            \includegraphics[width=0.4\textwidth]{img/chapitre2/figure6/eels_spectra_haadf-sc.png}
            \label{fig-chapitre2-real-eels-haadf}
        }\\
        \subfigure[Les spectres pour les trois positions]{
            \includegraphics[]{img/chapitre2/figure6/eels_spectra.pdf}
            \label{fig-chapitre2-real-eels-spectra}
        }\\
        %
        \subfigure[Bande A]{% $\mathrm{O-K}$
            \includegraphics[width=0.3\textwidth]{img/chapitre2/figure6/eels_spectra_band_410.png}
            \label{fig-chapitre2-real-eels-bandA}
        }
        \hspace*{0.5cm}
        %
        \subfigure[Bande B ($\mathrm{La-M}_{4, 5}$)]{
            \includegraphics[width=0.3\textwidth]{img/chapitre2/figure6/eels_spectra_band_1001.png}
            \label{fig-chapitre2-real-eels-bandB}
        }
        \hspace*{0.5cm}
        %
        \subfigure[Bande C ($\mathrm{Nd-M}_{4, 5}$)]{
            \includegraphics[width=0.3\textwidth]{img/chapitre2/figure6/eels_spectra_band_1465.png}
            \label{fig-chapitre2-real-eels-bandC}
        }

        \caption{Exemple réel d'image \gls{eels}. La Figure~\protect\subref{fig-chapitre2-real-eels-haadf} représente l'image \gls{haadf} de l'acquisition et trois positions en rouge, bleu et vert. Les spectres acquis en ces trois positions sont représentés à la figure~\protect\subref{fig-chapitre2-real-eels-spectra}. Enfin, les données spatiales à trois niveaux de perte d'énergie (notées A, B et C à la figure~\subref{fig-chapitre2-real-eels-spectra}) sont représentées aux figures~\subref{fig-chapitre2-real-eels-bandA} à \subref{fig-chapitre2-real-eels-bandC}. Les bandes B (resp. C) correspondent à la signature $\mathrm{La-M}_{4, 5}$ (resp. $\mathrm{Nd-M}_{4, 5}$) tandis que la bande A ne correspond à aucune signature.
            \protect\label{fig-chap2-eels-real-spectra-figure}}
    \end{figure}

    Plus précisément, l'effet d'un seuil ne se limite pas au pic puisque un biais positif persiste après le seuil. La courbe peut alors être décomposé en la somme d'un fond décroissant (généralement modélisé par une exponentielle décroissante) et de divers sauts, comme le montre la \cref{fig-decroissance-spectre}. Cette modélisation est la base des techniques de cartographie vues en \cref{sec-exploitation-eels}.

    \begin{figure*}
        \centering
        \subfigure[\label{fig-seuil-a}]{\includegraphics[]{img/chapitre2/figure6-bis/seuils.pdf}}
        \hspace{1em}
        \subfigure[\label{fig-seuil-b}]{\includegraphics[]{img/chapitre2/figure6-bis/seuils-2.pdf}}
        \caption{Visualisation des effets de seuils successifs sur le spectre. La figure~\subref{fig-seuil-a} représente la forme d'un spectre constitué de trois seuils. Le fond décroissant est mis en évidence ainsi que le biais rémanent associé à chaque saut. La figure~\subref{fig-seuil-b} représente les effets isolés de chacun des trois seuils. Nous y distinguons le pic caractéristique de l'élément ainsi que le biais suivant chaque pic.
            \protect\label{fig-decroissance-spectre}}
    \end{figure*}


    \paragraph*{Caractéristiques spatiales} En microscopie, comme pour beaucoup de systèmes d'imagerie, la caractéristique principale est la résolution. En microscopie \gls{stem}, celle-ci est principalement limitée par la taille de la sonde électronique (typiquement 1~nm) et par les aberrations sphériques (un correcteur permet de descendre à 0.1~nm). D'autre part, la résolution est choisie par l'expérimentateur en fonction de la distance parcourue par la sonde entre deux acquisitions. Cependant, en imagerie \gls{eels}, la taille de l'image excède rarement $10^5$ pixels pour limiter le temps d'acquisition et éviter de détériorer l'échantillon (comme nous verrons plus loin). Par conséquent, deux situations apparaissent~:
    \begin{itemize}
        \item l'expérimentateur étudie des structures spatiales étendues (typiquement 100~nm) et est obligé de limiter la résolution, les images sont alors basse-résolution (cf. \cref{fig-chap2-haadf-ex-a}),
        \item l'expérimentateur étudie des réseaux atomiques très localisés (typiquement quelques nanomètres) et la résolution est limité par l'instrument, les images sont alors haute-résolution (cf. \cref{fig-chap2-haadf-ex-b}).
    \end{itemize}
    Enfin, il faut noter que la structure spatiale est plus ou moins visible selon le canal considéré dans l'image \gls{eels}. Par exemple, la \cref{fig-chapitre2-real-eels-bandA} correspond à une zone du spectre où aucun contraste n'apparaît clairement, il en résulte que l'image est fortement bruitée. A contrario, les images situées sur des seuils particuliers du spectre sont moins bruitées et mettent clairement en évidence la position des éléments concernés, comme pour les \cref{fig-chapitre2-real-eels-bandB,fig-chapitre2-real-eels-bandC}. Ainsi, l'utilisateur peut avoir une première idée de la cartographie des éléments étudiés, même si celle-ci est biaisée à cause de la contribution des seuils précédents (certains atomes apparaissent dans la \cref{fig-chapitre2-real-eels-bandA} du fait du seuil précédent la bande A, visible vers 500eV en \cref{fig-chapitre2-real-eels-spectra}). L'exploitation des données \gls{eels} sera approfondi dans la \cref{sec-exploitation-eels}.

    \paragraph*{Nature du bruit} Les sources de bruit sont multiples en imagerie \gls{eels}, mais le type de bruit le plus attendu est poissonien. Cela modélise tant la probabilité qu'un électron du faisceau subisse un certain nombre de diffusions inélastique au sein de l'échantillon~\cite[Section~4.1.1]{egerton2011electron} que la probabilité qu'une cellule du capteur reçoive un électron sur une période donnée. \`A cela s'ajoute un bruit gaussien dû à l'électronique d'acquisition.
    %
    En pratique, la nature du bruit est expérimentalement complexe à déterminer.
    %
    A ce bruit d'acquisition vient s'ajouter les instabilités spatiales de l'échantillon : celui-ci dérive au cours de l'acquisition dû à des variations de température, des mouvements d'air~\cite{zobelli2019spatial}. Ces déplacements ne sont pas visibles pour des images basse-résolution mais deviennent critiques à des échelles atomiques puisque le réseau atomique initialement aligné apparaît déformé, comme le montre la Figure~\ref{fig-drift}.
    %
    Enfin, des incertitudes de positionnement du faisceau s'ajoutent au cours de l'acquisition.

    \begin{marginfigure}
    	\centering
    	\includegraphics[width=0.7\textwidth]{img/chapitre2/figure7/drift-sc.png}
    	\caption{Un exemple de défaut en haute résolution : la dérive de l'échantillon. L'échantillonnage se fait ligne par ligne. On observe que, dans ce cas, l'échantillon dérivait sur la gauche, puis vers le haut. Il en résulte une déformation notable et préjudiciable du réseau.}
            %L'échantillon est un grenat de bismuth et de fer.}
    	\label{fig-drift}
    \end{marginfigure}


    \paragraph*{Redondance spectrale} L'imagerie \gls{eels} est un cas particulier d'imagerie hyperspectrale, dans les deux cas, il s'agit d'une acquisition spectroscopique. Les images rencontrées dans de nombreuses modalités incluant l'imagerie hyperspectrale et \gls{eels} sont connues pour être hautement corrélées spectralement~\cite{dobigeon_linear_2016, bioucas2012hyperspectral, dobigeon2012spectral}. Il en résulte que les données représentées dans un espace de dimension $N$ (où $N$ est le nombre de canaux) n'occupent pas indifféremment l'espace mais sont localisés dans une variété de dimension réduite. Celle-ci est non-linéaire par défaut, et une approximation linéaire est généralement réalisée par \gls{pca}. Une représentation de cette propriété est donnée à la \cref{fig-correlation}. L'analyse par \gls{pca} a plusieurs intérêts parmi lesquels le débruitage  et la réduction de la dimension des données \gls{eels}. En effet, les valeurs propres associées à l'\gls{pca} (représentée en \cref{fig-sub-pca-eigs}) sont généralement séparées en deux groupes suivant leur comportement : les premières de puissance\footnote{Il faut comprendre la puissance d'une composante principale comme la valeur propre associée à cette composante.} significativement supérieures et les autres de puissance semblable correspondant à la variance du bruit. Il en résulte que les données peuvent être considérées comme faible rang. Il s'agit là d'une propriété hautement intéressante qui sera utilisée dans la suite du manuscrit. Relevons enfin que les données peuvent être fortement débruités en seuillant l'\gls{pca}, c'est-à-dire en ne conservant que les premières composantes principales, puis en effectuant une rétro-projection.

    \begin{figure}[htb]
    	\centering
    	\includegraphics[]{img/chapitre2/figure8/redondance.pdf}
    	\caption{Une illustration de la redondance spectrale des données \gls{eels}. En considérant des spectres RVG (donc à 3 composantes), il est possible de les représenter dans l'espace. Or, les acquisitions (représentées ici par des points) n'occupent pas l'espace entier, mais se localisent plutôt à proximité d'une variété représentée en gris. L'analyse en composante principale permet d'obtenir une approximation \emph{linéaire} de la variété. Cet exemple considère une variété de dimension 2 dans un espace de dimension 3, mais cela se généralise pour des spectres de dimension $N$. Dans ces conditions, la variété est de dimension inférieure à $N$.
            \protect\label{fig-correlation}}
    \end{figure}


    \begin{figure}[htbp]
    	\centering
    	\subfigure[Evolution des valeurs propres associées à l'\gls{pca}]{
    		\includegraphics[]{img/chapitre2/figure9/PCA_eigs.pdf}
    		\label{fig-sub-pca-eigs}}
    	\\
    	\subfigure[Composante 0]{
    		\includegraphics[width=0.28\textwidth]{img/chapitre2/figure9/PCA_map_0-sc.png}
    		\label{fig-sub-pca-comp0}
    	}\vspace{20pt}
    	\subfigure[Composante 1]{
    		\includegraphics[width=0.28\textwidth]{img/chapitre2/figure9/PCA_map_1.png}}\vspace{20pt}
    	\subfigure[Composante 2]{
    		\includegraphics[width=0.28\textwidth]{img/chapitre2/figure9/PCA_map_2.png}}\\
    	\subfigure[Composante 3]{
    		\includegraphics[width=0.28\textwidth]{img/chapitre2/figure9/PCA_map_3.png}}\vspace{20pt}
    	\subfigure[Composante 4]{
    		\includegraphics[width=0.28\textwidth]{img/chapitre2/figure9/PCA_map_4.png}}\vspace{20pt}
    	\subfigure[Composante 5]{
    		\includegraphics[width=0.28\textwidth]{img/chapitre2/figure9/PCA_map_5.png}
    		\label{fig-sub-pca-comp5}
    	}\\
    	%
    	\def\comp{mean}
    	\subfigure[Spectre moyen]{
    		\includegraphics[]{img/chapitre2/figure9/PCA_spectra_1.pdf}
    		\label{fig-sub-pca-mean}}%
    	%
    	\def\comp{comp0}%
    	\subfigure[Première composante - spectre]{
    		\includegraphics[]{img/chapitre2/figure9/PCA_spectra_2.pdf}
    		\label{fig-sub-pca-spectrum-0}}
    	%
    	\caption{L'analyse par \gls{pca} de données \gls{eels}. La figure~\subref{fig-sub-pca-eigs} représente l'évolution des valeurs propres associées à l'\gls{pca}. On observe, entre autres, que seules les 5 premières composantes sont vraiment significatives. Les figures \subref{fig-sub-pca-comp0} à \subref{fig-sub-pca-comp5} représentent la projection des données sur les spectres des composantes principales les plus puissantes. Le spectre moyen donné à la figure~\subref{fig-sub-pca-mean} est soustrait aux données avant d'appliquer l'\gls{pca}. Il en résulte que les spectres associés aux composantes principales sont dès lors bruitées, comme pour la composante 0 à la figure~\subref{fig-sub-pca-spectrum-0}.
            \protect\label{fig-ACP}} 
    \end{figure}


    %
    \section{Exploitation des données \glsentryshort{eels}}\label{sec-exploitation-eels}

    Comme expliqué précédemment, les données \gls{eels} présentent un intérêt tout particulier en cartographie d'éléments chimiques. Cette technique permet non seulement de cartographier la distribution spatiale des éléments, mais également de détecter des structures fines correspondant à des structures électroniques locales. Les images représentant la répartition de ces éléments sont appelés \emph{cartes d'abondances}. Deux méthodes sont classiquement utilisées~: la cartographie par séparation de composantes spectrales et les techniques de démélange.

    \subsection{Cartographie par séparation de composantes spectrales}

    Comme expliqué en \cref{sec-prop-eels}, un spectre \gls{eels} contient la contribution de plusieurs seuils successifs et d'un fond décroissant. Pour cartographier l'élément associé à un seuil, une méthode naïve consiste à intégrer le spectre autour du seuil pour chaque position spatiale et de représenter l'image obtenue. Néanmoins, le fond décroissant et la rémanence de seuils précédents biaise ce résultat, faisant apparaître des structures erronées dans les cartes d'abondance.

    Pour pallier ce problème, le fond décroissant et la rémanence des seuils précédents devront être soustrait avant d'intégrer le seuil d'intérêt. Pour illustrer cette méthode, considérons le spectre affiché en \cref{fig-carto-separation-a}, correspondant à une position spatiale donnée. Une zone de régression est choisie en amont du seuil (tout en restant proche du seuil) et une régression exponentielle est effectuée (cf~\cite[Section~4.4]{egerton2011electron} pour plus de détails). La courbe estimée est alors soustraite pour obtenir un spectre corrigé, c'est-a-dire que le seuil étudié n'est plus influencé ni par le fond décroissant, ni par les seuils précédents. L'abondance du composé pour cette position spatiale est donc calculée en sommant le spectre dans une zone centrée sur le seuil d'intérêt. En effectuant cette opération pour toutes les positions spatiales, nous pouvons reconstituer la carte d'abondance de la \cref{fig-carto-separation-b}. \`A noter que les paramètres de la régression dépendent de la position spatiale.

    Afin d'effectuer la cartographie de plusieurs composantes présentes dans l'échantillon, l'expérimentateur doit alors exécuter cette opération pour tous les seuils présents dans le spectre. Cette technique est simple, mais chronophage. C'est pourquoi des méthodes non-supervisées comme le démélange ont été élaborées pour simplifier l'exploitation des données.

    \begin{figure*}
        \centering
        \subfigure[\label{fig-carto-separation-a}]{
            \includegraphics[]{img/chapitre2/figure11/separation.pdf}}\hspace{1em}
        \subfigure[\label{fig-carto-separation-b}]{
            \includegraphics[width=5cm]{img/chapitre2/figure11/regression-sc.png}}
        \caption{Illustration de la méthode de cartographie par séparation de composantes spectrales. \protect\subref{fig-carto-separation-a} régression pour une position particulière. Une régression exponentielle est effectuée sur la zone de régression, puis est soustraite au spectre pour obtenir le spectre corrigé. L'abondance est finalement calculée en intégrant le spectre autours du pic d'intérêt. \protect\subref{fig-carto-separation-b} résultat en effectuant cette action sur tous les pixels (l'élément cartographié est $\mathrm{La-M}_{4, 5}$). A noter que les paramètres de la régression dépendent de la position spatiale.
            \protect\label{fig-carto-separation}}
    \end{figure*}

    \subsection{Cartographie par démélange}

    Nous l'avons dit plus haut, l'imagerie \gls{eels} est un cas particulier d'imagerie hyperspectrale. L'analyse de telles données consiste également à cartographier l'abondance d'un élément particulier dans une scène (une prise de vue aérienne, par exemple, on parle alors de \emph{remote sensing}). Une hypothèse classique consiste à considérer la zone étudiée comme une mixture de peu d'éléments, si bien que le spectre observé en une position spatiale est un mélange de spectres élémentaires. Le problème de cartographie consiste alors à estimer conjointement les spectres élémentaires et leur proportion pour chaque position spatiale. Ce problème inverse est appelé \emph{démélange hyperspectral}~\cite{bioucas2012hyperspectral} (ou unmixing). Puisque l'estimation conjointe des paramètres nécessite la résolution d'un problème non-convexe, on lui préfère une résolution sous-optimale mais convexe. Les spectres élémentaires sont alors estimés en premier lieu (à l'aide d'algorithmes comme VCA~\cite{nascimento2005vertex} ou SISAL~\cite{bioucas2009variable}), puis les cartes d'abondance sont restituées a posteriori (l'algorithme SUNSAL~\cite{bioucas2010alternating} est un exemple d'algorithme pour cette seconde étape).

    Cette technique a été appliquée avec succès en imagerie EELS~\cite{dobigeon2012spectral, Dobigeon_ELSEVIER_2016} et est prometteuse pour analyser efficacement les données \gls{eels}.


    %
    \section{L'acquisition d'échantillons sensibles : problématiques et stratégies}\label{sec-ech-sensibles}

    \paragraph{Problématique des échantillons sensibles} Pour améliorer la qualité de l'acquisition (et par là même celle de son exploitation), l'expérimentateur doit augmenter la dose totale d'électrons délivrée à chaque position de l'échantillon. Pour cela, il peut agir sur la vitesse des électrons du faisceau ou sur la durée d'exposition par position spatiale. Cependant, augmenter la dose accroît la détérioration subie par l'échantillon~\cite{egerton2004radiation}. Cela est particulièrement problématique pour les échantillons sensibles tels que les tissus biologiques. \emph{Il en résulte un compromis entre une qualité d'image satisfaisante et la préservation de l'échantillon d'étude}. Pour illustrer cela, un exemple d'échantillon détérioré par une concentration trop importante d'énergie est donné à la \cref{fig-echantillon-deteriore}. Dès lors, \emph{des stratégies sont nécessaires afin d'améliorer la qualité du spectre-image sans augmenter la quantité d'énergie délivrée à l'échantillon}.

    \begin{figure}[t]
        \centering
        \subfigure[\label{fig-echantillon-deteriore-a}]{%
            \includegraphics[width=0.3\textwidth]{img/chapitre2/figure12/sequentialScan.png}}\hspace{1em}
        \subfigure[\label{fig-echantillon-deteriore-b}]{%
            \includegraphics[width=0.3\textwidth]{img/chapitre2/figure12/randomScan.png}}
        \caption{\protect\subref{fig-echantillon-deteriore-a} Exemple d'échantillon détérioré par un faisceau trop énergétique. L'acquisition est réalisée ligne par ligne en commençant au pixel supérieur gauche. \`A un certain stade de l'acquisition, une structure apparaît. Celle-ci est censée être ronde, mais au lieu de cela, plus rien n'est observée quelques lignes après le début du signal. En effet, l'énergie du faisceau était trop grande et la structure a été détruite. A cela s'ajoute que l'acquisition ligne par ligne accumule trop d'énergie dans des positions successives, ce qui dégrade encore davantage l'échantillon. \protect\subref{fig-echantillon-deteriore-b} Acquisition d'une structure semblable avec une acquisition aléatoire et une énergie par pixel égale. La structure n'est pas détériorée puisque l'énergie est davantage répartie au cours de l'échantillonnage.
            \protect\label{fig-echantillon-deteriore}}
    \end{figure}

    \paragraph{Acquisition ligne par ligne v.s. aléatoire} Une première solution pour réduire la détérioration de l'échantillon à énergie fixée consiste à échantillonner de manière aléatoire uniforme. En effet, l'acquisition ligne par ligne est très simple à implémenter, mais cette modalité détériore particulièrement l'échantillon puisqu'elle accumule des doses d'énergie sur des pixels voisins. Pour pallier ce problème, des travaux récents ont mis au point un obturateur coupant le faisceau au cours de l'acquisition~\cite{beche2016development, tararan2016random}, rendant le chemin d'acquisition hautement paramétrable. Ainsi, un parcours aléatoire de l'échantillon permet de visiter successivement des positions spatiales éloignées, ce qui répartie la dose d'électron sur l'ensemble de l'échantillon. Cela est mis en évidence en \cref{fig-echantillon-deteriore-b} puisque la structure spatiale est totalement visible pour une acquisition aléatoire tandis que l'échantillonnage ligne par ligne détruit la structure (cf \cref{fig-echantillon-deteriore-a}). Cette technique permet donc d'augmenter l'énergie délivrée à l'échantillon pour une détérioration égale, et donc d'améliorer la qualité de l'image.

    \paragraph{Correction de la dérive de l'échantillon} Comme expliquée en \cref{sec-prop-eels}, l'échantillon n'est pas stable au cours de l'acquisition puisque des gradients de température et des mouvements d'air le font dériver. Dans le cas d'une acquisition ligne par ligne, cela se manifeste par une déformation du réseau atomique, comme le montre la \cref{fig-drift}. Dans le cas d'un échantillonnage aléatoire, cela détériore grandement la résolution spatiale, particulièrement pour les échantillons à échelle atomique. Pour corriger cela, une méthode consiste à extraire $N$ sous-acquisitions partielles de l'acquisition aléatoire complète. Chacune de ces acquisition partielle est alors complété indépendamment à l'aide d'une méthode de reconstruction (cf \cref{ch-chapter_2}) et le mouvement de l'échantillon est estimé. Les échantillons associés à chaque sous-acquisition sont alors recalés pour compenser la dérive. Cette méthode est appelée \emph{multi-trame} et a été implémentée avec succès en imagerie \gls{haadf} et \gls{eels}~\cite{zobelli2019spatial}. Cette méthode améliore sensiblement la qualité de l'image.

    \paragraph{Débruitage v.s. inpainting} La méthode la plus simple pour limiter la détérioration de l'échantillon consiste à réduire la dose d'électron et à débruiter les données en post-traitement. Cependant, réduire l'exposition peut être d'intérêt limité puisque l'acquisition est trop dégradée pour atteindre des performances de débruitage satisfaisantes, tout particulièrement dans le cas où les structures analysées sont fines.
    %
    Si bien qu'une alternative consiste à conserver une dose d'électrons élevée, mais à ne visiter que peu de positions spatiales. Cette technique rendue possible par l'acquisition aléatoire ne fournie pourtant qu'une image partielle et une méthode de reconstruction est nécessaire afin de compléter les pixels manquants. A cette fin, le \cref{ch-chapter_2} fera le point sur les techniques de reconstruction d'image et sur leur utilisation en microscopie. Cette approche est un domaine de recherche très actif en microscopie \gls{stem}~\cite{beche2016compressed,stevens2014potential} et \gls{sem}~\cite{anderson2013sparse} entre autres.
    %
    Les deux stratégies proposées ont à la fois des avantages et des inconvénients. D'une manière générale, une acquisition à faible dose fournit des informations spatiales plus riches tandis que les données partiellement acquises ont un contenu spectral plus riche. Déterminer quelle approche est la meilleure n'est pas trivial et des études récentes ont comparées ces deux approches~\cite{trampert2018ultramicroscopy} en se basant sur des expériences réalisées sur des images synthétiques et réelles.
    
    % Parler aussi des travaux concernant le choix d'un bon chemin d'acquisition. -> chap 2



    \section{\'Etat des installations au \glsentryshort{lps}}

    Avant d'étudier les techniques de reconstruction dans le cadre du traitement de l'image, il convient de faire le point sur les techniques déjà testées au sein de l'équipe STEM du \gls{lps}.

    La position de la sonde électronique sur l'échantillon est pilotée par des bobines magnétiques. Le mode d'acquisition aléatoire a été implémenté à l'aide d'un module de balayage basé sur FPGA (Field-Programmable Gate Array) contrôlant directement les bobines de balayage~\cite{zobelli2019spatial}. L'ordre des pixels est mélangé aléatoirement dans une phase initiale, puis ce chemin aléatoire est chargé dans un générateur de chemin. Pour chaque pixel, le faisceau illumine l’échantillon uniquement pendant le temps requis pour l’acquisition du signal, temps durant lequel la caméra est en mode d’exposition. Le faisceau d’électrons est coupé pendant le déplacement grâce à un obturateur de faisceau électrostatique. Ce système est actuellement installé sur deux microscopes \gls{stem} :
    \begin{itemize}
        \item un VG-HB501 avec une résolution de l’ordre du nm (cf. \cref{fig-LPS-micro-a}),
        \item un Nion UltraSTEM 200 équipé d’un correcteur d’aberrations sphériques qui permet d’atteindre une résolution de l’ordre de 0,1 nm (cf. \cref{fig-LPS-micro-b}).
    \end{itemize}

    \begin{figure*}
        \centering  % 0.8, 0.5
        \subfigure[\label{fig-LPS-micro-a}]{\includegraphics[width=0.5\textwidth]{img/chapitre2/figure13/VG.jpg}}
        \hspace{1em}
        \subfigure[\label{fig-LPS-micro-b}]{\includegraphics[width=0.4\textwidth]{img/chapitre2/figure13/NION.jpg}}
        \caption{Microscopes en service au \gls{lps} sur lesquels le mode d'acquisition aléatoire est implémenté. \subref{fig-LPS-micro-a} VG-HB501. \subref{fig-LPS-micro-b} Nion UltraSTEM 200.
            \protect\label{fig-LPS-micro}}
    \end{figure*}

    Se basant sur ce système, une approche multi-frame corrigeant la dérive de l'échantillon et améliorant la qualité de l'image a été envisagée~\cite{zobelli2019spatial}. Cependant, cette technique est encore très peu utilisée car lourde d'un point de vue expérimental. En pratique, la dérive est limitée en laissant l'échantillon se stabiliser après insertion dans le microscope et en s'assurant que l'acquisition soit suffisamment rapide. Enfin, dans le cas d'un échantillonnage ligne par ligne où l'échantillon dérive uniformément, il est également possible de corriger le défaut par déformation géométrique du réseau.

    % Figure

    Le débruitage du spectre-image a été utilisé pour améliorer la qualité de l'acquisition, en particulier pour des temps d'exposition faibles ou des échantillons sensibles.  La technique la plus couramment utilisée en spectroscopie \gls{eels} consiste à appliquer une \gls{pca} et à ne conserver que les composantes les plus puissantes (cette technique a déjà été présentée aux \crefrange{sec-prop-eels}{sec-exploitation-eels}). D'autres approches ont été testées lors d'un stage étudiant en 2019, mais elles restent encore très peu utilisées.

    Enfin, les techniques de reconstruction sont encore assez marginales dans l'équipe et ont été utilisées en cathodoluminescence~\cite{zobelli2019spatial} et dans le cadre de cette thèse (échantillon biologique sur VG, NNO/LAO sur UltraStem). Des essais ont été menés \textit{in situ} (c'est-à-dire en chauffant l'échantillon et en observant l'évolution), mais les résultats sont encore peu probants.






%    \begin{subappendices}
%        \section{Quelques notions sur l'\glsentryshort{pca}} % (fold)
%        \label{app-pca}
%
%            Normal equation
%            \begin{equation}
%            \label{eq:normal}
%                \sum_{n=1}^{N} 1 / n \approx \ln(N)
%            \end{equation}
%
%            full width equation
%
%            \begin{fullwidth}
%                \begin{equation}
%                \label{eq:normal}
%                    \sum_{n=1}^{N} 1 / n \approx \ln(N)
%                \end{equation}
%            \end{fullwidth}
%
%            \subsection{subsection name} % (fold)
%            \label{app:sub:subsection_name}
%
%            % subsection subsection_name (end)
%
%        % section section_name (end)
%    \end{subappendices}

% chapter chapter_1 (end)


% ---

\chapter{Etat de l'art}
\dochaptoc
\label{ch-chapter_2}

%
\section{Formulation du problème d'inpainting}

\subsection{Présentation générale}

Un grand nombre de méthodes ont émergées au cours de la dernière décennie afin de reconstruire une image avec une grande précision en ne disposant que d'une acquisition partielle. Plus généralement, beaucoup de problèmes inverses ont été intensément étudiés sous le prisme de l'acquisition comprimée. L'acquisition comprimée est un framework général qui fournit des méthodes de restauration avec des garanties théoriques pour les problèmes inverses linéaires sous-déterminés. Ces travaux d'Emmanuel Candès, Justin Romberg, Terence Tao et David Donoho~\cite{candes2006near, candes2006stable, donoho2006compressed} ont révolutionné le domaine du traitement du signal. Ceux-ci ont démontré qu'une image acquise avec une fréquence d'échantillonnage inférieure à celle de Nyquist pouvait être restaurée de manière exacte sous certaines conditions (dont une spécifiant que les données doivent être parcimonieuses dans une certaine base). Le paradigme de l'acquisition comprimée requiert que les données soient projetées sur $n$ sous-espaces aléatoires, avec $n$ très petites devant la taille des données. Cette technique a été appliqué avec succès dans de nombreux domaines incluant l'IRM~\cite{boyer_algorithm_2014}, l'imagerie ultrasonique~\cite{quinsac_bayesian_2011}, l'astronomie~\cite{bobin_compressed_2008} ou la tomographie~\cite{binev2012compressed, jacob2019MM, jacob2018MM} en microscopie. %
%
Ces résultats ont rendu populaire les problèmes inverses visant à compléter une image à partir d'une acquisition spatialement incomplète. Ces méthodes sont appelées techniques de \emph{complétion} ou encore d'\emph{inpainting}\footnote{Le terme d'inpainting a été introduit par les travaux de Bertalmio \textit{et al.}~\cite{bertalmio2000image} en référence aux techniques de restauration en art. En effet, leur modèle est basé sur l'observation du travail effectué par les artistes employés par les musés pour restaurer les vielles peintures.}. Il s'agit d'un domaine de recherche très actif en microscopie \gls{stem}~\cite{beche2016compressed,stevens2014potential} et \gls{sem}~\cite{anderson2013sparse} entre autres.

L'inpainting regroupe cependant deux situations distinctes nécessitant des techniques distinctes, comme le montre la \cref{fig-inpainting}. D'une part, l'information peut être fortement structurée, comme c'est le cas en retouche photographique où un élément (personne, objet) doit être supprimé d'une prise de vue~\cite{criminisi2004region}. Les \crefrange{fig-inpainting-a}{fig-inpainting-c} montrent un exemple de complétion pour des données structurées. Nous retrouvons également dans cette classe la correction de données aberrantes apparues suite à un dysfonctionnement du capteur~\cite{zhang2013hyperspectral}. D'autre part, l'acquisition peut être volontairement lacunaire pour des raisons de compression ou de préservation de l'échantillon (comme dans notre cas), l'information est alors répartie. Les \crefrange{fig-inpainting-d}{fig-inpainting-f} montrent un exemple de cette situation.

\begin{figure}
    \centering
    \subfigure[\label{fig-inpainting-a}]{
        \includegraphics[width=0.25\textwidth]{img/chapitre3/figure1/initial-2.png}}\hspace{1em}
    \subfigure[\label{fig-inpainting-b}]{
        \includegraphics[width=0.25\textwidth]{img/chapitre3/figure1/mask-2.png}}\hspace{1em}
    \subfigure[\label{fig-inpainting-c}]{
        \includegraphics[width=0.25\textwidth]{img/chapitre3/figure1/final-2.png}}\\
    %
    \subfigure[\label{fig-inpainting-d}]{
        \includegraphics[width=0.25\textwidth]{img/chapitre3/figure2/initial.png}}\hspace{1em}
    \subfigure[\label{fig-inpainting-e}]{
        \includegraphics[width=0.25\textwidth]{img/chapitre3/figure2/mask.png}}\hspace{1em}
    \subfigure[\label{fig-inpainting-f}]{
        \includegraphics[width=0.25\textwidth]{img/chapitre3/figure2/final.png}}
    \caption{Exemple d'inpainting extrait de~\protect\cite{peyre2011numerical}. \protect\subref{fig-inpainting-a} Exemple d'image destinée à la retouche photographique. \protect\subref{fig-inpainting-b} Les pixels composant la grille sont enlevés de l'image. \protect\subref{fig-inpainting-c} Résultat après inpainting : la grille est enlevée de l'image. \protect\subref{fig-inpainting-d} Exemple d'acquisition complète. \protect\subref{fig-inpainting-e} La même image acquise partiellement. \protect\subref{fig-inpainting-f} Résultat après inpainting.
        \protect\label{fig-inpainting}} 
\end{figure}

La technique d'inpainting est généralement associée à la reconstruction d'images 2D, mais elle s'étend au-delà pour les images multi-dimensionnelles dont une partie des voxels (l'équivalent des pixels pour une image multi-dimensionnelle) sont manquants.
%
La stratégie d'acquisition partielle décrite à la \cref{sec-ech-sensibles} s'inscrit dans ce contexte. En effet, la sonde ne visite qu'une partie de l'échantillon en suivant un chemin généralement aléatoire. Il en résulte des données spatialement sous-échantillonées que les techniques d'inpainting peuvent compléter.
%
Notons également que ce schéma d'acquisition spatial ne s'accompagne pas d'un sous-échantillonnage spectral puisque, pour chaque position spatiale, le spectromètre \gls{eels} sépare simultanément toutes les pertes d'énergie conduisant à une acquisition complète du spectre. Il en résulte que les données issues d'une telle stratégie sont \emph{fortement structurées}.


\subsection{Problème direct et inverse}

Notons $\gls{X}\in\mathbb{R}^{\gls{M}\times\gls{P}}$ les données \gls{eels} inconnues à retrouver où \gls{P} est le nombre de pixels et \gls{M} est le nombre de canaux. %
%
Comme expliqué à la \cref{sec-ech-sensibles}, faire l'acquisition complète du spectre-image \gls{X} n'est pas toujours possible dû à la détérioration introduite par le faisceau d'électron sur l'échantillon. Pour empêcher cela, la zone d'intérêt ne peut être échantillonné qu'à certaines positions spatiales. Ainsi, les spectres complets sont acquis en \gls{N} positions parmi \gls{P}, il en résulte un rapport d'échantillonnage $\gls{r}=\gls{N}/\gls{P}$ et l'ensemble des index des \gls{N} positions spatiales visitées est noté \gls{I}. %

D'autre part, nous avons vu à la \cref{sec-prop-eels} que le bruit attaché aux données est le mélange de plusieurs contributions dont le modèle statistique diffère. Il en résulte que quantifier chacune de ces contribution est complexe en pratique et les modèles classiquement retenu dans la littérature sont les bruits poissonien~\cite{egerton2011electron, mevenkamp2015poisson, stevens2018apl} et gaussien~\cite{stevens2014potential, binev2012compressed}. Dans ce manuscrit, nous choisirons de modéliser la détérioration des données avec un bruit indépendant, additif, gaussien et blanc. Deux raisons principales expliquent ce choix :
\begin{itemize}
    \item puisque les différentes composantes de ce mélange sont difficile à quantifier, nous préférons utiliser le bruit gaussien plus simple,
    \item des expériences ont été réalisées avec un bruit poissonien seul~\cite{Monier2020SuppNum} et aucune différence notable n'a été observée par rapport à un modèle gaussien.
\end{itemize}
De plus, si une composante poissonienne émerge particulièrement, des techniques de stabilisation de variance comme la transformée de Anscombe~\cite{anscombe1948transformation} convertissent le bruit poissonien en un bruit gaussien.

Finalement, la matrice observée $\gls{Y}\in\mathbb{R}^{\gls{M}\times\gls{N}}$ peut être décrite par le modèle direct suivant :
\begin{equation}
    \gls{Y} = \gls{X}_{\gls{I}} + \gls{B}
\end{equation}
où $\gls{X}_{\gls{I}}$ est la matrice réalisée en concaténant les colonnes de \gls{X} indexées par \gls{I} et \gls{B} est un terme résiduel associé à l'erreur de modèle et le bruit d'acquisition. Les éléments de \gls{B} sont supposés être indépendants et identiquement distribués selon une loi gaussienne centrée d'écart type \gls{sig}.

Le problème de reconstruction consiste à restituer une image complète (et possiblement débruitée) \gls{X} à partir de \gls{Y}. Cependant, cette tâche est mal posée puisque le nombre de paramètres \gls{P}\gls{M} est supérieur au nombre d'observations \gls{N}\gls{M} et la suite de ce chapitre étudiera les approches possibles pour résoudre ce problème inverse.

%
\section{Les différentes classes d'inpainting}

Les techniques de reconstructions classiquement rencontrées dans la littérature vont être présentées dans cette section. Elles ont été classées en quatre sections : les techniques d'interpolation, les techniques de \gls{mc} régularisés, les techniques par diffusion et les techniques par patch.

\subsection{Les techniques d'interpolation}

\paragraph{Présentation du problème d'interpolation} Définissons un ensemble de points $(p_k)_{1\leq k\leq K}$ appartenant à un espace Euclidien de dimension $n$ (typiquement $\mathbb{R}^n$) et un ensemble de valeurs associées $(f(p_k))_{1\leq k\leq K}$ avec $f:X\rightarrow \mathbb{R}$ la fonction à interpoler. Le problème d'interpolation consiste à déterminer les valeurs prises par $f$ en un ensemble de points $(q_k)_{1\leq k\leq K}$ quelconques de $X$.
%
Ce problème est défini que l'information soit structurée ou non. Dans notre cas, nous devons considérer nos données comme un cube dont certains voxels sont manquant. L'étape d'interpolation consiste à compléter les valeurs prises par l'image en ces points.
%
Ce problème est facile en 1D et reste simple à condition que les points $(p_k)_{1\leq k\leq K}$ soient régulièrement disposés dans l'espace. Dans le cas d'un ensemble de points irrégulièrement répartis (comme c'est le cas ici), l'espace doit être découpé en éléments de base, puis interpolée sur ces éléments. Par exemple, en 2D, la surface peut être découpée en triangles. Afin d'introduire cette technique qui servira de référence dans la suite, le diagramme de Voronoi et la triangulation de Delaunay vont être succinctement définis.


\begin{marginfigure}
    \centering
    \includegraphics[]{img/chapitre3/figure3/Voronoi.pdf}
    \caption{Supperposition d'un diagramme de Voronoi (en rouge) et de sa triangulation de Delaunay (en noir). Le réseau de points est en vert.}
    \label{fig-voronoi}
\end{marginfigure}
\paragraph{Diagrammes de Voronoi et triangulation de Delaunay} Afin d'introduire les techniques d'interpolation, deux outils mathématiques simples et complémentaires seront nécessaires. Le premier, appelé diagramme de Voronoi~\cite{cazals2006delaunay}, est défini pour un ensemble de points comme suit.
\begin{mydef}[Diagramme de Voronoi]
    Soit $X$ un espace métrique de distance $d$ et $(p_k)_{1\leq k\leq K}$ un ensemble de $K$ points de $X$. La cellule de Voronoi $R_k$ associée au point $p_k$ est l'ensemble des points de $X$ plus proches de $p_k$ que de tout autre point $p_j$ pour $j$ différent de $k$. En d'autre termes, si $d(p_k, p_j)$ désigne la distance entre $p_k$ et $p_j$, nous avons
    \[R_k=\{x\in X | d(x, p_k) \leq d(x, p_j) \ \forall j\neq k\}.\]
    Le diagramme de Voronoi est définit comme l'ensemble des cellules $(R_k)_{1\leq k\leq K}$.
\end{mydef}
Le diagramme de Voronoi se visualise bien si $X$ est un espace euclidien de dimension 2 et cette notion est illustrée en rouge à la~\cref{fig-voronoi}. Une problématique communément associée à ce graphe est la \emph{triangulation} qui consiste à découper un plan en une collection de triangles. En effet, la triangulation de Delaunay~\cite{cazals2006delaunay} d'un ensemble discret de points est le graphe dual du diagramme de Voronoi et se définit comme suit.
\begin{marginfigure}
    \centering
    \includegraphics[]{img/chapitre3/figure3/Delaunay.pdf}
    \caption{Superposition d'un ensemble de points (en vert), de sa triangulation de Delaunay (en noir) et des cercles circonscrit à chaque triangle (en gris).}
    \label{fig-delaunay}
\end{marginfigure}
\begin{mydef}[Triangulation de Delaunay]
    Soit $P=(p_k)_{1\leq k\leq K}$ un ensemble de points appartenant à un espace Euclidien de dimension $n$. Une triangulation de Delaunay $\mathrm{DT(P)}$ est une triangulation telle qu'aucun point de $P$ ne se trouve dans l'hypersphère circonscrite d'un simplex de $\mathrm{DT(P)}$.
\end{mydef}
Notons qu'une hypersphère circonscrite (resp. un simplex) est la généralisation du cercle circonscrit (resp. du triangle) en dimension quelconque. La triangulation de Delaunay est illustrée en noir à la \cref{fig-voronoi} et les cercles circonscrit associés à chaque triangle sont mis en évidence pour un ensemble de point réduit à la \cref{fig-delaunay}. Ces deux outils vont servir à présent pour définir les techniques d'interpolation classiques.


\paragraph{Interpolation par \Glsentrylong{ppv}} L'interpolation par \gls{ppv} est la technique d'interpolation la plus simple et la moins coûteuse. Elle consiste à associer à chaque point à interpoler la valeur prise par $f$ au point échantillonné $p_{k}$ le plus proche. Cela consiste à associer $f(p_{k})$ à tout point $x$ appartenant à la cellule de Voronoi de $p_{k}$. Cette méthode, bien que très peu coûteuse, est aussi la moins efficace puisque la fonction interpolée est constante par morceaux, ce qui entraîne une erreur de reconstruction élevée. Cela est particulièrement visible sur l'exemple d'interpolation \gls{ppv} donné aux figures \ref{fig-interpolation-a} et \ref{fig-interpolation-d} puisque les cellules de Voronoi apparaissent clairement sur l'image tandis que la fonction interpolée est constante par morceaux. %
\begin{marginfigure}
    \centering
    \includegraphics[]{img/chapitre3/figure3/Voronoi_Natural.pdf}
    \caption{Illustration de la méthode par plus proches voisins naturels. Le diagramme de Voronoi associé à l'ensemble de point (en vert) est affiché (en rouge). Lorsque le nouveau point (en orange) est ajouté, une cellule (en gris) est ajoutée au diagramme.}
    \label{fig-natural-weight}
\end{marginfigure}
Pour lisser davantage la fonction interpolée, une solution consiste à associer au point à interpoler $x$ une pondération des valeurs prises par l'ensemble de points $(p_k)_{1\leq k\leq K}$, c'est-à-dire
\begin{equation}\label{eq-weighted-interp}
    \hat{f}(x) = \sum_{k=1}^K w_k(x) f(p_k)
\end{equation}
où $w_k(x)$ est le poids de $x$ associé au point échantillonné $p_k$ (la somme des poids vaut 1). Une approche classique consiste à évaluer le diagramme de Voronoi de $(p_k)_{1\leq k\leq K}$ (en rouge à la \cref{fig-natural-weight}), puis une deuxième fois en ajoutant $x$ (la cellule supplémentaire est en gris à la \cref{fig-natural-weight}). Si l'on note $A_k$ l'aire de l'intersection entre cette nouvelle cellule et la cellule précédemment associée à $p_k$ et $A$ l'aire de la nouvelle cellule associée à $x$, alors on pose $w_k(x)=A_k/A$. Cette approche est appelée interpolation par plus proches voisins naturels~\cite{sibson1981interpreting,cazals2006delaunay}. Elle donne de meilleurs résultats que l'interpolation par plus proche voisins, mais elle est aussi plus lourde d'un point de vue calculatoire.

\paragraph{Interpolation pour des ordres supérieurs} Comme expliqué plus haut, l'interpolation dans le cas de points non-uniformément répartis est réalisé en ajustant une fonction d'ordre fixe sur chaque simplex issu de la triangulation de Delaunay. Ainsi, l'interpolation \gls{ppv} ajuste une fonction constante par morceau sur les sommets du simplex. Des ordres supérieurs peuvent être utilisés, comme l'interpolation linéaire ou cubique en 1D correspondant respectivement à des fonctions d'ordre 1 et 2. En 2D, cela consiste à ajuster des plans ou des paraboles aux sommets du triangle. En plus grande dimension, cela est plus complexe et est réalisé entre autre par interpolation barycentrique~\cite{hormann2014barycentric}. Les figures~\ref{fig-interpolation-b}, \ref{fig-interpolation-c}, \ref{fig-interpolation-e} et \ref{fig-interpolation-f} permettent de visualiser l'ajustement de plans dans le cas linéaire et de paraboles dans le cas cubique.
  
\begin{figure}[h]
    \centering
    \subfigure[\label{fig-interpolation-a}\gls{ppv}]{
        \includegraphics[width=0.29\textwidth]{img/chapitre3/figure4/nearest.pdf}}\hspace{1em}
    \subfigure[\label{fig-interpolation-b}Interpolation linéaire]{
        \includegraphics[width=0.29\textwidth]{img/chapitre3/figure4/linear.pdf}}\hspace{1em}
    \subfigure[\label{fig-interpolation-c}Interpolation cubique]{
        \includegraphics[width=0.29\textwidth]{img/chapitre3/figure4/cubic.pdf}}\\
    \subfigure[\label{fig-interpolation-d}\gls{ppv} - surface]{
        \includegraphics[width=0.29\textwidth]{img/chapitre3/figure4/surf_nearest.pdf}}\hspace{1em}
    \subfigure[\label{fig-interpolation-e}Interpolation linéaire - surface]{
        \includegraphics[width=0.29\textwidth]{img/chapitre3/figure4/surf_linear.pdf}}\hspace{1em}
    \subfigure[\label{fig-interpolation-f}Interpolation cubique - surface]{
        \includegraphics[width=0.29\textwidth]{img/chapitre3/figure4/surf_cubic.pdf}}
    \caption{Un ensemble de point (en rouge) est aléatoirement tiré et des valeurs leur sont associées. La triangulation de Delaunay est représentée en noire. La fonction est interpolée sur le carré unité. \protect\subref{fig-interpolation-a} Interpolation \gls{ppv}. \protect\subref{fig-interpolation-b} Interpolation bilinéaire. \protect\subref{fig-interpolation-c} Interpolation bicubique. Les fonctions interpolées sont également représentées dans le même ordre sous forme de surfaces aux figures~\subref{fig-interpolation-d} à \subref{fig-interpolation-f} (la résolution spatiale est diminuée pour des raisons d'affichage).
        \protect\label{fig-interpolation}}
\end{figure}

\paragraph{Avantages et inconvénients} Les méthodes d'interpolation ont pour principal avantage leur faible complexité et leur rapidité, ce qui en font des méthodes populaires en reconstruction en ligne~\cite{sibson1981interpreting, cazals2006delaunay, trampert2018ultramicroscopy}. Néanmoins, elles ne sont pas robustes puisqu'elles s'ajustent aux valeurs disponibles elles-mêmes corrompues. Elles ne permettent pas d'introduire plus d'information a priori que l'ordre de la fonction à interpoler.

\subsection{Les techniques de \glsentrylong{mc} régularisés}

Une approche \emph{variationelle} en inpainting est une méthode calculant l'image reconstruite en minimisant une fonction objectif appelée \emph{fonctionnelle}. En particulier, une technique de \gls{mc} régularisée fournit une image reconstruite \gls{Xh} en résolvant le problème de minimisation suivant~\cite[Section~6.3]{boyd2004convex}
\begin{equation}
    \gls{Xh} = \arg\min_{\gls{X}} \mathcal{L}(\gls{X}_{\gls{I}}, \gls{Y}) + \lambda \mathcal{R}(\gls{X})
\end{equation}
où $\mathcal{L}(\cdot, \gls{Y})$ est le terme d'attache aux données et où l'opérateur $\mathcal{R}$ est une régularisation. Le scalaire $\lambda$ permet d'ajuster l'importance de la régularisation par rapport au terme d'attache aux données. Cette formulation est liée à l'estimateur du maximum a posteriori puisque la formule de Bayes donne la fonction de vraisemblance $f(\gls{X}|\gls{Y}) \propto f(\gls{Y}|\gls{X}) f(\gls{X})$. En considérant l'opposé de la log-vraissemblance, l'estimée est l'image \gls{X} minimisant la fonction
\begin{equation}
    -\log f(\gls{X}|\gls{Y}) \propto
    \underbrace{-\log f(\gls{Y}|\gls{X})}_{\mathcal{L}(\gls{X}_{\gls{I}}, \gls{Y})}
    \underbrace{- \log f(\gls{X})}_{\mathcal{R}(\gls{X})}.
\end{equation}
Il en résulte que l'opérateur $\mathcal{L}$ est choisi en fonction du modèle statistique du bruit tandis que la régularisation dépend de l'information a priori disponible pour \gls{X}. Ces méthodes gèrent donc mieux la connaissance du bruit que les techniques d'interpolation et sont ainsi plus robustes. En particulier, le terme d'attache aux données pour un bruit additif gaussien blanc est une fonction coût quadratique $||\gls{Y}-\gls{X}_{\gls{I}}||_F^2$. Notons encore que les problèmes de \gls{mc} régularisés conviennent que l'information soit structurée ou non et qu'elles sont également utilisés pour la complétion en se basant sur une triangulation, comme c'est le cas en ajustement de surface~\cite{zhong2016surface,cazals2006delaunay}.

Un choix particulièrement classique pour la régularisation est la norme quadratique $\mathcal{R}(\gls{X}) = ||\gls{X}||_F^2$, on parle alors de régularisation de Tikhonov. L'avantage de cette forme est que la solution est  donnée par une expression mathématique directe, ne nécessitant qu'une inversion de matrice. La probabilité associée à \gls{X} est gaussienne centrée et promeut des données d'amplitude faible.

Les techniques de \gls{mc} régularisés sont également d'un intérêt particulier dans le cas de données  parcimonieuses, c'est-à-dire dont la proportion d'entrées non nulles est très faible. Dans ce cas, la régularisation idéale est la pseudo-norme $\ell_0$ \footnote{La pseudo-norme $\ell_0$ de \gls{X} vaut le nombre d'entrées non-nulle de \gls{X}. Il ne s'agit pas d'une norme puisque $||\alpha\gls{X}||_0 = ||\gls{X}||_0$ pour tout scalaire $\alpha$ non nul.} puisque celle-ci contraint le nombre d'éléments non-nuls. Malheureusement, résoudre ce problème d'estimation est très compliqué en pratique puisque la norme $\ell_0$ n'est pas faiblement convexe et on lui préfère généralement une relaxation convexe, comme la norme $\ell_1$ définie par $||\gls{X}||_1 = \sqrt{\sum |\gls{X}_i|}$. Cette méthode est souvent appelée Lasso~\cite{tibshirani1996regression} de l'acronyme \emph{Least Absolute Shrinkage and Selection Operator}. Enfin, il faut noter qu'utiliser la norme $\ell_1$ comme régularisation biaise le résultat. Pour corriger cela, des travaux ont proposé de résoudre le problème inverse en deux temps. D'abord, la technique Lasso est appliquée afin de déterminer les entrées non-nulles de l'estimée. Ensuite, une régression des moindres carrés est réalisée sur ce support. Cette méthode en deux temps est appelée post-LS (pour Least Square) ou encore \emph{refitting}~\cite{belloni2013least, lederer2013trust, deledalle2017clear}.

Utiliser la norme de $\gls{X}$ comme régularisation n'est pas toujours adaptée, en particulier lorsque le problème est sous-déterminé (i.e. où le nombre d'observations est inférieur au nombre de pixels). En effet, le terme d'attache aux données ne contraint pas les pixels manquants et en minimisant $||\gls{X}||$, ceux-xi prendraient des valeurs nulles. L'idée consistant à pénaliser les fortes valeurs prises par $\gls{X}$ peut être étendue de plusieurs façons. L'une d'entre elle consiste à pénaliser $\mathcal{A}\gls{X}$ où $\mathcal{A}$ est un opérateur linéaire. Ainsi, une alternative très populaire en traitement du signal consiste plutôt à pénaliser le gradient de l'image $\nabla\gls{X}$, conduisant à une régularisation $\mathcal{R}(\gls{X})=||\nabla\gls{X}||_F^2$, on parle aussi d'énergie de Sobolev. Le gradient est ainsi minimisé et une image lissée est restituée, comblant ainsi les régions sous-échantillonnées. De même, la régularisation $\ell_1$ peut être couplée avec un opérateur linéaire $\mathcal{A}$ pour favoriser la parcimonie du signal dans un cas particulier, par exemple :
\begin{itemize}
    \item si $\mathcal{A}$ est un changement de base telle que $\mathcal{A}\gls{X}$ soit parcimonieuse, la régularisation $||\mathcal{A}\gls{X}||_1$ est adaptée,
    \item si le gradient de l'image est calculée, la régularisation $||\nabla \gls{X}||_1$ appelée \gls{tv} promeut une image ayant peu de contours (l'image résultante est constante par morceaux).
\end{itemize}


\subsection{Les techniques par diffusion}

La diffusion est un processus physique très intuitif tendant à équilibrer les différences de concentration au sein d'un fluide sans créer ou détruire de masse. Ce phénomène est régi par l'équation de diffusion suivante
\begin{equation}
    \frac{\partial u}{\partial t} \triangleq \dot{u}(x, y, t) = \mathrm{div} (D(x, y)\cdot \nabla u)
\end{equation}
où $u$ est la concentration, $D$ est le coefficient de diffusion et $\nabla$ est l'opérateur gradient. En traitement de l'image, on identifie la concentration avec la valeur en niveau de gris prise en une position particulière. Si le coefficient de diffusion est constant sur toute l'image, on parle de diffusion \emph{isotropique} (ou linéaire), sinon, on parle de diffusion \emph{anisotropique} (ou non-linéaire).

La technique de diffusion la plus simple est la diffusion isotropique en débruitage, conduisant au problème aux dérivées partielles suivant
\begin{align}
&\dot{u} = D \cdot \Delta u\\
&u(\cdot, t=0) = y
\end{align}
où $y$ est l'image bruitée. Ce problème est équivalent à un filtrage avec un noyau gaussien d'écart-type $\sqrt{2t}$\footnote{L'image resultante en diffusion n'est pas l'image $u(\cdot, t=\infty)$ puisque celle-ci est constante (la diffusion tend à égaliser les niveaux de gris). Il faut choisir un instant $t^*$ où arrêter la trajectoire et l'image resultante est $u(\cdot, t^*)$.}~\cite{weickert1998anisotropic}. Le problème de cette technique est que la diffusion introduit un flou sur les contours de l'image. C'est pourquoi Perona et Malik~\cite{perona1990scale} ont proposé la diffusion anisotropique pour préserver les contours de l'image. Le coefficient de diffusion est diminué au niveau des contours tandis qu'il reste élevé au sein de zones homogènes. Le problème résultant s'écrit
\begin{align}
    &\dot{u}(x, y, t) = \mathrm{div} (D(|\nabla u|^2)\cdot \nabla u)\\
    & D(s) = \frac{1}{1+s^2/\lambda^2} \quad \text{pour $\lambda > 0$}
\end{align}

La technique de diffusion ne suffit cependant pas pour l'inpainting puisque la structure n'est pas propagée et un transport de matière doit être ajouté. Les travaux de Bertalmio \textit{et al.}~\cite{bertalmio2000image} ont été pionniers dans ce domaine et ont été à l'origine du terme \emph{inpainting}. S'inspirant des techniques de restauration en art, ils ont envisagé une technique par transport de matière propageant l'information le long de lignes de niveau (les lignes reliant les points de l'image ayant le même niveau de gris) dans le cas où celle-ci est structurée. Un terme de diffusion anisotropique était ajouté afin d'éviter que les lignes de niveau ne se croisent. D'autres techniqiues basée sur la variation totale on suivi~\cite{shen2002mathematical, chan2001nontexture} mais le principe fondamental consiste à propager la structure.

Ces techniques sont très bien adaptées aux images où l'information est très structurée, mais elles ne conviennent pas lorsque l'information est répartie. En effet, ces techniques reposent sur la propagation de contour. Dans le cas où les données sont réparties, aucun contour ne peut être propagé. C'est pourquoi nous n'utiliserons pas ces méthodes dans ce manuscrit.


\subsection{Les techniques par patch}

Une extension des méthodes variationelles exploite la redondance spatiale dans l'image, on parle alors de \emph{méthode par patch}. Elles constituent un ensemble de méthodes très populaires et performantes dont l'intérêt n'a cessé de grandir ces dernières décennies afin de résoudre des problèmes inverses comme le débruitage, l'inpainting ou la déconvolution.

Ces méthodes sont à opposer aux techniques dites \emph{locales} où une valeur est corrigée en ne la comparant qu'avec son voisinage, comme lorsqu'une image est convoluée avec un masque gaussien en débruitage. Le premier exemple de méthode \emph{non-locale} a été l'algorithme de débruitage Non-Local Mean~\cite{buades2005non} qui recherche des patchs semblables dans l'image afin de moyenner les pixels centraux. D'autres techniques plus évoluées en débruitage ont suivi, parmi lesquelles Block Matching and 3D filtering (BM3D)~\cite{dabov2007image} et Non Local Bayes~\cite{lebrun2013nonlocal}. Cependant, cette approche ne peux s'appliquer à l'inpainting seulement si l'information est structurée. Par exemple, l'algorithme \gls{ebi}~\cite{criminisi2004region} reconstruit des images partiellement corrompues en remplaçant de manière itérative les patchs manquants par le patch entier lui ressemblant le plus dans son voisinage. Des résultats ont également exprimés Non-Local Means sous forme variationelle, permettant ainsi des applications en reconstruction d'images dont l'information est répartie~\cite{peyre2008non, unni2018non, arias2009variational, yang2012nonlocal}.

Pour imposer la redondance spatiale, des algorithmes performants cherchent à représenter les patchs de l'image de manière parcimonieuse à l'aide d'\emph{atomes} contenus dans un \emph{dictionnaire}. Ces patchs caractéristiques sont appris conjointement aux données reconstruites, on parle alors de méthode par \gls{ad}. Une formulation générique de cette technique peut s'écrire comme le problème de \gls{mc} régularisé suivant~\cite{mairal2009online}
\begin{equation}
    \begin{aligned}
    (\mathbf{D}^*,\mathbf{A}^*) = &\arg\min_{(\mathbf{D}, \mathbf{A})}
    \frac{1}{2} || \mathbf{R}(\gls{Y}) - \mathbf{D A}||_F^2 + \lambda  || \mathbf{A} ||_1\\
    &\text{tel que } || \mathbf{D}_k ||_2 = 1 \quad \forall k\\
    \end{aligned}
\end{equation}
où $\mathbf{D}$ et $\mathbf{A}$ sont les matrices contenant respectivement les atomes et la décomposition parcimonieuse associé à chaque patch (on parle aussi de \emph{code}). $\mathbf{R}$, quant à lui, est l'opérateur permettant d'extraire les patchs des données et l'image reconstruite est obtenue par $\mathbf{R}^{-1}(\mathbf{D}^*\mathbf{A}^*)$. La contrainte permet de normaliser les atomes tandis que la régularisation $|| \mathbf{A} ||_1$ contraint le code à être parcimonieux. %
%
Cependant, la fonctionnelle à minimiser est non-convexe, rendant l'estimation conjointe du code et du dictionnaire compliqué, si bien que l'on préfère alterner le calcul du code et l'apprentissage du dictionnaire dont les formulations isolées sont convexes (mais le résultat est sous-optimal). C'est ainsi que l'algorithme de reconstruction wKSVD~\cite{mairal2008tip} alterne l'estimation des atomes par K-SVD~\cite{aharon2006k} et l'apprentissage du code par OMP~\cite{mallat1993matching, pati1993orthogonal} (la métrique de OMP est modifiée afin de mieux convenir aux images colorimétriques). L'algorithme ITKrMM~\cite{naumova2018fast, naumova2017dictionary} est un autre exemple d'algorithme fonctionnant sur ce principe, mais il a l'avantage d'être plus rapide puisqu'il estime le code à l'aide d'une opération de seuillage au lieu d'utiliser l'algorithme OMP plus lent. %
%
{\color{red}D'autres travaux proposent d'adopter une approche bayesienne afin d'estimer le code et le dictionnaire comme une densité de probabilité, l'algorithme très populaire en microscopie Beta-Process Factor Analysis (BPFA)~\cite{xing2012siam} initialement développé pour la reconstruction de données hyperspectrales. Ces techniques par \gls{ad} conviennent quelle que soit le type d'inpainting.}

Enfin, il faut bien mettre en avant que la reconstruction d'une image par \gls{ad} ne doit reposer sur les données corrompues \emph{uniquement}. Cette approche ne nécessitant aucune autre donnée est appelée \emph{méthode sans entraînement}. Cependant, il est aisé de reconstruire l'image dès lorsque le dictionnaire est disponible, si bien que les atomes sont parfois appris sur un jeu de données non-corrompues et utilisés ensuite pour la reconstruction, on parle alors de \emph{méthode par entraînement}. Cette approche est connue pour donner de bien meilleurs résultats, mais présente deux inconvénients majeurs :
\begin{itemize}
    \item cela nécessite un grand ensemble d'images propres (appelé données d'entraînement),
    \item le contenu spatial de l'image à reconstruire doit être similaire au contenu des données d'entraînement.
\end{itemize}
Les performances sont particulièrement dégradées si les structures sont différentes, tournées différemment ou d'échelle différente.



%
\section{Utilisation des techniques de reconstruction en microscopie}

Ce manuscrit se concentre sur la reconstruction d'images \gls{stem} spatialement sous-échantillonnées. Beaucoup de travaux ont étudié ce problème en employant différentes approches et modalités. Cette partie traite de ces travaux en les divisant en deux parties. La première considère les méthodes \emph{sans entraînement} qui reconstruisent les images à l'aide des seules données acquises. La seconde étudie les méthodes \emph{avec entraînement} qui s'appuient sur des données d'entraînement pour calibrer un opérateur utilisé ensuite en complétion de données.

\subsection{Méthodes sans entraînement}

L'interpolation \gls{ppv} est une solution simple et rapide, autorisant parfois de réaliser conjointement l'acquisition et la reconstruction en temps réel. Cependant, pour éviter que l'image reconstruite soit constante par morceau, on préfère pondérer les valeurs prises par les pixels voisins, comme expliqué à l'\cref{eq-weighted-interp}. La distance entre le pixel à interpoler et le voisin est inversée puis normalisée afin de former la pondération. Cette approche est utilisée en reconstruction d'images \gls{sem} dans~\cite{godaliyadda2018tci}, \gls{edx} dans~\cite{zhang2018reduced, hujsak2018high} et \gls{eels} dans~\cite{hujsak2018high}. L'interpolation par plus proches voisins naturels qui ajuste les poids en se basant sur une représentation de Voronoi est choisi comme alternative pour des images \gls{sem} dans~\cite{trampert2018ultramicroscopy}.

Les méthodes par \gls{mc} régularisés offrent généralement de meilleurs résultats que \gls{ppv} puisqu'elles contraignent l'image reconstruite à suivre un comportement prédéfini, généralement promue par une régularisation adaptée. Un choix classique puisque motivée par le paradigme de l'acquisition comprimée promeut la parcimonie dans une base adaptée, comme la régularisation $\ell_1$ utilisée pour des images de \gls{mfa} dans~\cite{han2018optimal}. Ce type de \gls{mc} régularisé sera noté $\ell_1-\gls{mc}$ dans le \cref{tab-litterature}. %
Dans le cas de structures périodiques (comme c'est le cas pour des images à échelle atomique), la base de Fourier ou la \gls{dct} peuvent être utilisées. Les auteurs de~\cite{stevens2018apl} ont ainsi proposé une méthode de reconstruction d'image \gls{haadf} reposant sur une transformée de Fourier seuillée, contraignant ainsi la parcimonie dans cette base périodique. La méthode proposée dans~\cite{beche2016development} utilise l'algorithme SPGL1~\cite{berg2008probing} afin de promouvoir la parcimonie dans la base \gls{dct} et reconstruire des images \gls{haadf}. De la même façon, cette régularisation peut être couplée avec une base d'ondelette pour reconstruire des images \gls{haadf} en ligne\cite{li2018compressed}. %
%
La régularisation \gls{tv} est aussi classiquement utilisée pour favoriser la reconstruction d'images constantes par morceaux, comme proposé en~\cite{han2018optimal} pour des données \gls{mfa}. La représentation \gls{dct} par block a été couplée avec la \gls{tv} en reconstruction d'images \gls{sem} dans~\cite{anderson2013sparse}.%



Les méthodes par patch offrent généralement de meilleurs performances puisqu'elles exploitent la redondance spatiale. En particulier, les techniques par \gls{ad} estiment conjointement les atomes constituant un dictionnaire et la représentation parcimonieuse associée. %
%
BPFA est très probablement la méthode par \gls{ad} la plus populaire dans la communauté de microscopie~\cite{xing2012siam}. Il a été appliqué pour la première fois sur des images \gls{haadf} à échelle atomique~\cite{stevens2013potential} et a été utilisé par la suite dans de nombreux travaux pour le même type de données~\cite{mucke2016practical,kovarik2016implementation}.
%
Les auteurs de BPFA ont également proposé la méthode Kruskal-factor analysis (KFA) comme une extension tensorielle de BPFA~\cite{stevens2017tensor}. KFA a été appliqué à la reconstruction d'images \gls{eels} en se basant sur une acquisition multiplexée d'un spectre-image~\cite{stevens2016mm}.
%
Enfin, l'algorithme EPLL (Expected-Patch Log-Likelihood) fait l'hypothèse que la distribution statistique des patchs suit un mélange de lois gaussiennes~\cite{zoran2011from}. Cependant, cet algorithme est particulièrement lent, si bien que les auteurs ont préférés une implémentation simplifiée mais accélérée appelée Fast EPLL (FEPLL) afin de reconstruire des images \gls{sem}~\cite{parameswaran2019accelerating}.
%
En plus des méthodes utilisée dans la communauté de microscopie, wKSVD~\cite{mairal2008tip} et ITKrMM~\cite{naumova2018fast,naumova2017dictionary} apprennent le dictionnaire à partir de données incomplètes sans faire d'hypothèse particulière sur la distribution des patchs. Il s'agit pourtant de méthodes de l'état de l'art et elles seront considéré dans la suite de l'étude.

Afin d'atteindre de meilleures performances avec une dose d'électron réduite, plusieurs chemins d'acquisition ont été proposés, comme l'échantillonnage aléatoire de lignes horizontales~\cite{kovarik2016implementation,han2018optimal}, l'échantillonnage mixe régulier-aléatoire~\cite{stevens2018apl}, les chemins en spirale~\cite{sang2017dynamic,li2018compressive,han2018optimal} ou enfin les chemins en forme de carré~\cite{han2018optimal}.
%
Ces résultats tendent à montrer que les performances optimales sont atteintes par des acquisitions semi-aléatoires, qui introduisent de l'aléatoire dans des structures régulières, ce qui évite les grands trous. %
%
Finalement, des améliorations conséquentes en reconstruction ont été rendus possibles par l'acquisition adaptative qui sélectionne le pixel à échantillonner à partir des données précédemment acquises. Dans~\cite{dahmen2016feature}, les auteurs proposent de faire un premier échantillonnage à bas SNR afin de localiser les contours de l'image. Une seconde acquisition à SNR élevé est ensuite effectuée sur ces contours seulement. Enfin, les parties lisses de la première image sont filtrées tandis que les contours sont remplis avec les pixels issues de la seconde acquisition. Un schéma d'acquisition adaptative alternatif proposé dans~\cite{dahmen2019adaptive} consiste à localiser de manière itérative des points d'intérêt à échantillonner. Les techniques d'acquisition adaptatives partielles avec entraînement sont présentées à la sous-section suivante.

\subsection{Méthodes avec entraînement}

Contrairement aux méthodes sans apprentissage qui reconstruisent l'image complète à partir des seules données acquises, les méthodes avec entraînement apprennent un opérateur en exploitant des données d'entraînement. Par exemple, l'algorithme GOAL apprend un dictionnaire en maximisant la parcimonie de la représentation des données d'apprentissage~\cite{hawe2013analysis}. Le dictionnaire estimé est ensuite utilisé afin de reconstruire les données de test. De manière similaire, EBI qui est initialement une technique sans apprentissage peut être adapté afin de tirer parti de données d'apprentissage. Pour cela, au lieu d'extraire le patch à copier du voisinage, comme c'est le cas dans l'implémentation conventionnelle de EBI, ce patch est choisi parmi un dictionnaire appris auparavant sur des images non-corrompues. Cette stratégie est suivie dans~\cite{trampert2018exemplar} pour reconstruire des données 3D en \gls{sem}. GOAL et la version avec apprentissage de EBI ont été appliqués dans~\cite{trampert2018ultramicroscopy} pour des images 2D en \gls{sem}, mais BPFA semblait donner de meilleurs résultats.

Les approches avec apprentissage peuvent aussi être envisagée afin de décider quelle position devrait être échantillonnée pour minimiser la distorsion après reconstruction. En effet, la position des pixels échantillonnés impacte grandement la qualité de la reconstruction lorsque les données sont sous-échantillonnées~\cite{trampert2018ultramicroscopy}. Pour améliorer cette qualité, l'algorithme SLADS (supervised learning approach for dynamic sampling) apprend une fonction (appelée réduction de distorsion espérée (RDE)) indiquant quelle position devrait être échantillonnée afin de réduire la distorsion au maximum\cite{godaliyadda2018tci}. Cette étape d'apprentissage repose sur une liste de caractéristiques et sur des données labellisées et a été utilisée pour dynamiquement échantillonner des images \gls{sem}.
%
Cette méthode a aussi été appliquée à des données \gls{edx} dans~\cite{zhang2018reduced}. Pour cela, un réseau de neurones convolutif classifie les spectres des données test et la fonction RDE est calculée simultanément pour tous les labels. %
%
Le papier~\cite{hujsak2018high} a ensuite modifié cette approche pour permettre des mélanges d'éléments en \gls{eels} et en \gls{edx}. Toutes ces approches requirent une technique de reconstruction rapide et l'interpolation pondérée \gls{ppv} a été utilisée.



%
\section{Motivations et contribution}

Dans cette étude, nous envisageons le schéma d'acquisition partielle afin de préserver l'échantillon en microscopie \gls{stem}-\gls{eels}. En particulier, une motivation consiste à réduire le temps de calcul associé à l'étape d'inpainting en vue de l'insérer dans un processus d'acquisition. L'expérimentateur devrait être capable de visualiser le spectre-image complet au cours de l'acquisition, ce qui requiert à la fois un temps de calcul réduit et une bonne précision. En plus de ce processus en ligne, l'expérimentateur devrait être capable de raffiner le spectre-image reconstruit a posteriori. Dans ces conditions, des algorithmes plus précis mais nécessitant plus de temps de calcul sont autorisés. 

\begin{table*}[t]
    \centering
    \bgroup
    \renewcommand{\arraystretch}{1.2}
    \begin{tabular}{>{\arraybackslash\centering}m{3cm}*{5}{c}}
        \toprule
        \multirow{2}*{Famille}& \multirow{2}*{Méthode}& \multicolumn{2}{c}{Travaux}& 
        \multirow{2}*{Temps d'exécution}& \multirow{2}*{Précision}\\
        %
        &&2D&3D&&\\
        %
        \midrule
        %
        \multirow{2}*{Interpolation} & \gls{ppv} & - & - & \plusfa[3] & \minusfa[2]\\
        %
        & Voisinage pondéré & \cite{sibson1981interpreting, cazals2006delaunay, trampert2018ultramicroscopy}&
        - & \plusfa[2] & \minusfa[1]\\
        %
        \midrule
        %
        \multirow{2}*{\gls{mc} régularisés}&
        $\ell_1$-\gls{mc} & \cite{han2018optimal,beche2016development,li2018compressed,anderson2013sparse}&
        & \plusfa & \plusfa\\
        %
        & TV-\gls{mc} & \cite{han2018optimal} & - & \plusfa[1] & \plusfa\\
        %
        %
        \midrule
        %
        \multirow{6}{3cm}{\centering Méthode par \gls{ad}}&
        BPFA & {\cite{stevens2014potential,trampert2018ultramicroscopy}} &
        \textit{\cite{xing2012siam}} & \minusfa[3] & \plusfa[3]\\
        %
        & EBI & \cite{trampert2018ultramicroscopy} & {\cite{trampert2018exemplar}} &
        \minusfa[1] & \plusfa[2]\\
        %
        & FEPLL & \textit{\cite{parameswaran2019accelerating}},\cite{hujsak2018high} &
        - & \minusfa[1] & \plusfa[2]\\
        %
        & wKSVD & - & \textit{\cite{mairal2008tip}} & \minusfa[2] & \plusfa[2]\\
        %
        &
        ITKrMM & \textit{\cite{naumova2018fast}} & \textit{\cite{naumova2017dictionary}}&
        \minusfa[1] & \plusfa[2]\\
        %
        & GOAL & \textit{\cite{hawe2013analysis}},\cite{trampert2018ultramicroscopy}&
        - & \minusfa[1] & \plusfa[2]\\
        %
        \bottomrule
    \end{tabular}
    \egroup
    \caption{Comparaison des méthodes proposées dans la littérature en microscopie pour le reconstruction d'images partiellement échantillonnées. Des références supplémentaires n'étant pas issues de la littérature en microscopie sont données en italique. Le temps d'exécution et la précision sont évaluées qualitativement en se basant sur les résultats des chapitres suivants.%
        \protect\label{tab-litterature}}
\end{table*}

Afin de déterminer l'approche à privilégier, le \cref{tab-litterature} résume l'état de l'art en microscopie réalisé dans la section précédente Les méthodes y sont notées en fonction de leur complexité et de leur précision et groupées selon trois grandes familles : l'interpollation, les \gls{mc} regularisés et les méthodes par \gls{ad}. Pour chaque méthode, les travaux issus de la littérature sont fournis et séparés selon que les données reconstruites soient des images 2D mono-bandes (e.g. \gls{haadf}) ou des spectre-images 3D (e.g. \gls{eels}).

Parmi les méthodes compatibles avec les données \gls{eels}, \gls{ppv} est rapide mais les performances en reconstruction sont généralement mauvaises tandis que les méthodes par \gls{ad} sont très efficaces mais sont très lourdes en temps de calcul, tout particulièrement lorsque des patchs 3D sont appris. Par conséquent, cet état de l'art met en évidence une lacune : aucune technique disponible peut reconstruire précisément un spectre-image spatialement sous-échantillonné suffisamment rapidement pour l'inclure dans un processus expérimental d'acquisition en ligne. D'autant que l'acquisition et la reconstruction rapide d'un spectre-image \gls{eels} n'a suscité que peu d'intérêt en comparaison de sa contrepartie hyperspectrale en observation de la Terre~\cite{zhang_hyperspectral_2014, chayes_pre_processing_2017, golbabaee_hyperspectral_2012, chen_inpainting_2012}.

Une alternative pour systématiquement reconstruire un spectre-image consiste à traiter les images 2D associés à chaque canal \emph{séparément} et \emph{en parrallèle}. Notons d'ailleurs que \gls{ppv} fonctionne de cette façon lorsqu'il reconstruit un spectre-image spatialement sous-échantillonné. Cependant, cette approche est sous-optimale puisque la tâche de reconstruction est censée être plus efficace en s'appuyant sur les données 3D complètes. Des stratégies similaires consisteraient à ne reconstruire que les images associées à un ou plusieurs canaux d'intérêt nécessaires pour la cartographie d'éléments. Cependant, cette approche serait aussi sous-optimale quand aucun a priori serait disponible concernant l'échantillon à observer.

Pour conclure, ni l'interpolation \gls{ppv}, ni les méthodes par \gls{ad} ne sont adaptés à la reconstruction précise en ligne et seulement les méthodes par \gls{mc} allient précision et charge calculatoire réduite. C'est donc ce type d'approche qui a été étudié dans la suite de ce manuscrit. Et puisque les performances des méthodes par \gls{mc} régularisés dépendent fortement de l'information disponible a priori, la reconstruction d'images \gls{eels} en basse résolution sera étudiée dans le \cref{ch-chapter_3} tandis que les images haute résolution le seront dans le \cref{ch-chapter_4}. Remarquons finalement que les méthodes par \gls{ad} sont les techniques les plus performantes disponibles et qu'elles conviennent parfaitement à l'étape de raffinement effectuée hors processus d'acquisition. Ces techniques particulièrement performantes ne seront pas étudiées par la suite (bien qune approche non-locale régularisée ait été envisagée) et ne servirons que de comparaison aux approches proposées.







%
% Contribs
%

% ----------------
\begin{fullwidth}
	\part{Inpainting rapide en EELS}
\end{fullwidth}


% ---
\chapter{Inpainting rapide d'images basse-résolution}
\label{ch-chapter_3}
\dochaptoc
%
\section{Contexte des données basses résolution}
Expliquer l'échelle des données BR. Donner quelques images pour se fixer les idées.

Revenir sur les informations a priori (images lisses, faible rang).

%
\section{La méthode SNN}

\subsection{Formulation}
Décrire les régularisation utilisées et les raisons de ce choix. Donner le problème d'optimisation.

\subsection{Estimation des paramètres de régularisation}
Parler du principe de discipancy, du choix itératif des paramètres.

%
\section{La méthode 3S}

\subsection{Formulation}
Formuler le problème, parler de l'acp, etc.

\subsection{Choix des poids}
Reprendre l'étude pour le choix des poids.

\subsection{Correction de l'ACP}
Parler de la correction par Stein et par la régression isotonique. Mettre la régression isotonique en annexe.

%
\section{Implémentation}

\subsection{Le framework FISTA}

\subsection{Application à S2N}

\subsection{Application à 3S}

\subsection{Etude de la complexité}

%
\section{Résultats sur des données synthétiques}

\subsection{Création de données synthétiques}

Modèle pour générer les données, choix des spectres et matrice d'abondance. Création du bruit.

\subsection{Métriques et mesures de vraisemblance}

Discuter des mesures de vraisemblance pour les expériences présentes dans la thèse.

\subsection{Sensibilité aux paramètres}

\subsection{Performances de reconstruction}
Vérifier quelle méthode est la meilleure.

Y introduire les performances d'algorithmes d'apprentissage par dictionnaire.

\subsection{Comparaison avec un scénario de débruitage}
Le point de vue ici est plus expérimental.

%
\section{Résultats sur des données réelles}
Présenter les matériaux.

%
\section{Conclusion}

\begin{subappendices}

	\section{L'estimateur de Stein et la régression isotonique}
	\lipsum[1]

\end{subappendices}



% ---
\chapter{Reconstruction rapide d'images haute-résolution}
\label{ch-chapter_4}
\dochaptoc
%
\section{Contexte des images à échelle atomique}

\subsection{Description des images haute-résolution}

\subsection{Parcimonie groupée dans une base choisie}
Présenter le principe de parcimonie groupée et faire l'étude afin de déterminer que DCT est la base ``optimale''.

%
\section{La méthode CLS}

\subsection{Formulation}

\subsection{Implémentation et complexité}

%
\section{La méthode DL-CLS}

\subsection{Principe de la méthode}
Décrire le principe : appliquer CLS pour effectuer une opération d'apprentissage de dictionnaire et de code. Ceux-ci serviront d'initialisation aux codes de dictionary learning.

\subsection{Implémentation}
Parler de la méthode de dico learning.

%
\section{Expériences sur des données synthétiques}

\subsection{Création de données synthétiques}
Insister sur le fait qu'on utilise des images ``semi-réelles'' car la structure spatiale est plus importante ici.

Génération des spectres-images semi-réel et synthétique

\subsection{Sensibilité aux paramètres}

\subsection{Performances de reconstruction}
Vérifier quelle méthode est la meilleure.

Y introduire les performances d'algorithmes d'apprentissage par dictionnaire.

\subsection{Comparaison avec un scénario de débruitage}
Le point de vue ici est plus expérimental.

%
\section{Résultats sur des images réelles}

%
\section{Méthodes n'ayant pas fonctionné}
Discuter ici du refitting et du non-local regularized qui n'ont pas marchés.

%
\section{Conclusion}

%%%%%%%%%%%%%%%%%%%%%%%%%%%%%%%%%%%%%%%%%%%%%%%%%%%%%%%%%%%%%%%%%%%%%%%%%%
% The back matter contains unnumbered chapters
% conclusion, french summary, bibliographies, indices, glossaries
%%%%%%%%%%%%%%%%%%%%%%%%%%%%%%%%%%%%%%%%%%%%%%%%%%%%%%%%%%%%%%%%%%%%%%%%%%

\backmatter

%!TEX root = ../main.tex

\chapter*{Conclusion} % (fold)
\label{ch:conclusion}

% chapter conclusion (end)

%!TEX root = ../main.tex

\chapter*{Résumé en français} % (fold)
\label{ch:french_resume}

    Ce travail...

% chapter french_resume (end)

\begin{fullwidth}
	\bibliographystyle{IEEEtran}
	\bibliography{bib/strings_all_ref,bib/biblio}
\end{fullwidth}



\begin{fullwidth}
    \printindex
\end{fullwidth}



\end{document}
